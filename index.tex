% Options for packages loaded elsewhere
\PassOptionsToPackage{unicode}{hyperref}
\PassOptionsToPackage{hyphens}{url}
\PassOptionsToPackage{dvipsnames,svgnames,x11names}{xcolor}
%
\documentclass[
  letterpaper,
  DIV=11,
  numbers=noendperiod]{scrreprt}

\usepackage{amsmath,amssymb}
\usepackage{iftex}
\ifPDFTeX
  \usepackage[T1]{fontenc}
  \usepackage[utf8]{inputenc}
  \usepackage{textcomp} % provide euro and other symbols
\else % if luatex or xetex
  \usepackage{unicode-math}
  \defaultfontfeatures{Scale=MatchLowercase}
  \defaultfontfeatures[\rmfamily]{Ligatures=TeX,Scale=1}
\fi
\usepackage{lmodern}
\ifPDFTeX\else  
    % xetex/luatex font selection
\fi
% Use upquote if available, for straight quotes in verbatim environments
\IfFileExists{upquote.sty}{\usepackage{upquote}}{}
\IfFileExists{microtype.sty}{% use microtype if available
  \usepackage[]{microtype}
  \UseMicrotypeSet[protrusion]{basicmath} % disable protrusion for tt fonts
}{}
\makeatletter
\@ifundefined{KOMAClassName}{% if non-KOMA class
  \IfFileExists{parskip.sty}{%
    \usepackage{parskip}
  }{% else
    \setlength{\parindent}{0pt}
    \setlength{\parskip}{6pt plus 2pt minus 1pt}}
}{% if KOMA class
  \KOMAoptions{parskip=half}}
\makeatother
\usepackage{xcolor}
\setlength{\emergencystretch}{3em} % prevent overfull lines
\setcounter{secnumdepth}{5}
% Make \paragraph and \subparagraph free-standing
\makeatletter
\ifx\paragraph\undefined\else
  \let\oldparagraph\paragraph
  \renewcommand{\paragraph}{
    \@ifstar
      \xxxParagraphStar
      \xxxParagraphNoStar
  }
  \newcommand{\xxxParagraphStar}[1]{\oldparagraph*{#1}\mbox{}}
  \newcommand{\xxxParagraphNoStar}[1]{\oldparagraph{#1}\mbox{}}
\fi
\ifx\subparagraph\undefined\else
  \let\oldsubparagraph\subparagraph
  \renewcommand{\subparagraph}{
    \@ifstar
      \xxxSubParagraphStar
      \xxxSubParagraphNoStar
  }
  \newcommand{\xxxSubParagraphStar}[1]{\oldsubparagraph*{#1}\mbox{}}
  \newcommand{\xxxSubParagraphNoStar}[1]{\oldsubparagraph{#1}\mbox{}}
\fi
\makeatother

\usepackage{color}
\usepackage{fancyvrb}
\newcommand{\VerbBar}{|}
\newcommand{\VERB}{\Verb[commandchars=\\\{\}]}
\DefineVerbatimEnvironment{Highlighting}{Verbatim}{commandchars=\\\{\}}
% Add ',fontsize=\small' for more characters per line
\usepackage{framed}
\definecolor{shadecolor}{RGB}{241,243,245}
\newenvironment{Shaded}{\begin{snugshade}}{\end{snugshade}}
\newcommand{\AlertTok}[1]{\textcolor[rgb]{0.68,0.00,0.00}{#1}}
\newcommand{\AnnotationTok}[1]{\textcolor[rgb]{0.37,0.37,0.37}{#1}}
\newcommand{\AttributeTok}[1]{\textcolor[rgb]{0.40,0.45,0.13}{#1}}
\newcommand{\BaseNTok}[1]{\textcolor[rgb]{0.68,0.00,0.00}{#1}}
\newcommand{\BuiltInTok}[1]{\textcolor[rgb]{0.00,0.23,0.31}{#1}}
\newcommand{\CharTok}[1]{\textcolor[rgb]{0.13,0.47,0.30}{#1}}
\newcommand{\CommentTok}[1]{\textcolor[rgb]{0.37,0.37,0.37}{#1}}
\newcommand{\CommentVarTok}[1]{\textcolor[rgb]{0.37,0.37,0.37}{\textit{#1}}}
\newcommand{\ConstantTok}[1]{\textcolor[rgb]{0.56,0.35,0.01}{#1}}
\newcommand{\ControlFlowTok}[1]{\textcolor[rgb]{0.00,0.23,0.31}{\textbf{#1}}}
\newcommand{\DataTypeTok}[1]{\textcolor[rgb]{0.68,0.00,0.00}{#1}}
\newcommand{\DecValTok}[1]{\textcolor[rgb]{0.68,0.00,0.00}{#1}}
\newcommand{\DocumentationTok}[1]{\textcolor[rgb]{0.37,0.37,0.37}{\textit{#1}}}
\newcommand{\ErrorTok}[1]{\textcolor[rgb]{0.68,0.00,0.00}{#1}}
\newcommand{\ExtensionTok}[1]{\textcolor[rgb]{0.00,0.23,0.31}{#1}}
\newcommand{\FloatTok}[1]{\textcolor[rgb]{0.68,0.00,0.00}{#1}}
\newcommand{\FunctionTok}[1]{\textcolor[rgb]{0.28,0.35,0.67}{#1}}
\newcommand{\ImportTok}[1]{\textcolor[rgb]{0.00,0.46,0.62}{#1}}
\newcommand{\InformationTok}[1]{\textcolor[rgb]{0.37,0.37,0.37}{#1}}
\newcommand{\KeywordTok}[1]{\textcolor[rgb]{0.00,0.23,0.31}{\textbf{#1}}}
\newcommand{\NormalTok}[1]{\textcolor[rgb]{0.00,0.23,0.31}{#1}}
\newcommand{\OperatorTok}[1]{\textcolor[rgb]{0.37,0.37,0.37}{#1}}
\newcommand{\OtherTok}[1]{\textcolor[rgb]{0.00,0.23,0.31}{#1}}
\newcommand{\PreprocessorTok}[1]{\textcolor[rgb]{0.68,0.00,0.00}{#1}}
\newcommand{\RegionMarkerTok}[1]{\textcolor[rgb]{0.00,0.23,0.31}{#1}}
\newcommand{\SpecialCharTok}[1]{\textcolor[rgb]{0.37,0.37,0.37}{#1}}
\newcommand{\SpecialStringTok}[1]{\textcolor[rgb]{0.13,0.47,0.30}{#1}}
\newcommand{\StringTok}[1]{\textcolor[rgb]{0.13,0.47,0.30}{#1}}
\newcommand{\VariableTok}[1]{\textcolor[rgb]{0.07,0.07,0.07}{#1}}
\newcommand{\VerbatimStringTok}[1]{\textcolor[rgb]{0.13,0.47,0.30}{#1}}
\newcommand{\WarningTok}[1]{\textcolor[rgb]{0.37,0.37,0.37}{\textit{#1}}}

\providecommand{\tightlist}{%
  \setlength{\itemsep}{0pt}\setlength{\parskip}{0pt}}\usepackage{longtable,booktabs,array}
\usepackage{calc} % for calculating minipage widths
% Correct order of tables after \paragraph or \subparagraph
\usepackage{etoolbox}
\makeatletter
\patchcmd\longtable{\par}{\if@noskipsec\mbox{}\fi\par}{}{}
\makeatother
% Allow footnotes in longtable head/foot
\IfFileExists{footnotehyper.sty}{\usepackage{footnotehyper}}{\usepackage{footnote}}
\makesavenoteenv{longtable}
\usepackage{graphicx}
\makeatletter
\newsavebox\pandoc@box
\newcommand*\pandocbounded[1]{% scales image to fit in text height/width
  \sbox\pandoc@box{#1}%
  \Gscale@div\@tempa{\textheight}{\dimexpr\ht\pandoc@box+\dp\pandoc@box\relax}%
  \Gscale@div\@tempb{\linewidth}{\wd\pandoc@box}%
  \ifdim\@tempb\p@<\@tempa\p@\let\@tempa\@tempb\fi% select the smaller of both
  \ifdim\@tempa\p@<\p@\scalebox{\@tempa}{\usebox\pandoc@box}%
  \else\usebox{\pandoc@box}%
  \fi%
}
% Set default figure placement to htbp
\def\fps@figure{htbp}
\makeatother

\usepackage{ctex}
\usepackage{amsthm,mathrsfs}
\usepackage{fvextra}
\DefineVerbatimEnvironment{Highlighting}{Verbatim}{breaklines,commandchars=\\\{\}}
\KOMAoption{captions}{tableheading}
\makeatletter
\@ifpackageloaded{bookmark}{}{\usepackage{bookmark}}
\makeatother
\makeatletter
\@ifpackageloaded{caption}{}{\usepackage{caption}}
\AtBeginDocument{%
\ifdefined\contentsname
  \renewcommand*\contentsname{Table of contents}
\else
  \newcommand\contentsname{Table of contents}
\fi
\ifdefined\listfigurename
  \renewcommand*\listfigurename{List of Figures}
\else
  \newcommand\listfigurename{List of Figures}
\fi
\ifdefined\listtablename
  \renewcommand*\listtablename{List of Tables}
\else
  \newcommand\listtablename{List of Tables}
\fi
\ifdefined\figurename
  \renewcommand*\figurename{Figure}
\else
  \newcommand\figurename{Figure}
\fi
\ifdefined\tablename
  \renewcommand*\tablename{Table}
\else
  \newcommand\tablename{Table}
\fi
}
\@ifpackageloaded{float}{}{\usepackage{float}}
\floatstyle{ruled}
\@ifundefined{c@chapter}{\newfloat{codelisting}{h}{lop}}{\newfloat{codelisting}{h}{lop}[chapter]}
\floatname{codelisting}{Listing}
\newcommand*\listoflistings{\listof{codelisting}{List of Listings}}
\makeatother
\makeatletter
\makeatother
\makeatletter
\@ifpackageloaded{caption}{}{\usepackage{caption}}
\@ifpackageloaded{subcaption}{}{\usepackage{subcaption}}
\makeatother

\usepackage{bookmark}

\IfFileExists{xurl.sty}{\usepackage{xurl}}{} % add URL line breaks if available
\urlstyle{same} % disable monospaced font for URLs
\hypersetup{
  pdftitle={AI驱动Zotero智慧文献集成},
  pdfauthor={文探007:张睿、兰聪、黄俊豪},
  colorlinks=true,
  linkcolor={blue},
  filecolor={Maroon},
  citecolor={Blue},
  urlcolor={Blue},
  pdfcreator={LaTeX via pandoc}}


\title{AI驱动Zotero智慧文献集成}
\author{文探007:张睿、兰聪、黄俊豪}
\date{2025-03-21}

\begin{document}
\maketitle

\renewcommand*\contentsname{Table of contents}
{
\hypersetup{linkcolor=}
\setcounter{tocdepth}{2}
\tableofcontents
}

\bookmarksetup{startatroot}

\chapter{简介}\label{ux7b80ux4ecb}

本项目旨在构建一套集``文献抓取---智能筛选---数据存储---信息提取---成果展示''为一体的自动化文献管理系统。其核心功能包括:

\begin{itemize}
\item
  \textbf{自动采集}:系统首先通过 CrossRef
  批量采集与指定主题相关的文献。
\item
  \textbf{自动筛选}:AI
  模型会对每篇文献进行关键字打分,进一步评估语义相关性,仅保留高关联度的文章。
\item
  \textbf{自动归档}:高质量文献将被保存到 Zotero
  进行结构化存储和标签管理,确保文献信息完整有序。
\item
  \textbf{自动解读}:系统通过调用 AI 大模型,从 Zotero
  数据库中自动提取文献的关键信息并进行深入解读,并生成统一格式的
  Markdown 文档。
\item
  \textbf{自动集成}:利用 Quarto
  渲染生成直观、阅读流畅的网页展示界面,为用户呈现极致的文献浏览体验。
\end{itemize}

总之,本项目通过 AI
大模型整合文献采集、筛选、归档、解读和集成的自动化,可以成为您科研的好帮手。

\bookmarksetup{startatroot}

\chapter{技术内幕}\label{ux6280ux672fux5185ux5e55}

这一部分,我们将解释 \texttt{AI-Paper}
系统的工作原理,主要包括以下几个方面:

\section{采集文献}\label{ux91c7ux96c6ux6587ux732e}

这一步,系统首先通过 CrossRef
批量采集与指定主题相关的文献。用户可以通过设置关键词、时间范围等参数,来获取特定领域的文献数据。

\begin{Shaded}
\begin{Highlighting}[]
\ImportTok{import}\NormalTok{ openai}
\ImportTok{import}\NormalTok{ requests}
\ImportTok{import}\NormalTok{ yaml}
\ImportTok{import}\NormalTok{ time}
\ImportTok{import}\NormalTok{ logging}
\ImportTok{import}\NormalTok{ json}
\ImportTok{import}\NormalTok{ os}
\ImportTok{import}\NormalTok{ textwrap}

\CommentTok{\# {-}{-}{-}{-}{-}{-}{-}{-}{-}{-}{-}{-}{-}{-}{-}{-}{-}{-}{-}{-}{-}{-}{-}{-}{-}{-}{-}{-}{-}{-}{-}{-}{-}{-}{-}{-}{-}{-}{-}{-}{-}{-}{-}{-}{-}{-}{-}{-}{-}{-}{-}{-}{-}{-}{-}{-}{-}{-}{-}{-}}
\CommentTok{\# 日志配置:将日志级别设为 INFO,并统一日志格式}
\CommentTok{\# {-}{-}{-}{-}{-}{-}{-}{-}{-}{-}{-}{-}{-}{-}{-}{-}{-}{-}{-}{-}{-}{-}{-}{-}{-}{-}{-}{-}{-}{-}{-}{-}{-}{-}{-}{-}{-}{-}{-}{-}{-}{-}{-}{-}{-}{-}{-}{-}{-}{-}{-}{-}{-}{-}{-}{-}{-}{-}{-}{-}}
\NormalTok{logging.basicConfig(level}\OperatorTok{=}\NormalTok{logging.INFO, }\BuiltInTok{format}\OperatorTok{=}\StringTok{\textquotesingle{}}\SpecialCharTok{\%(asctime)s}\StringTok{ {-} }\SpecialCharTok{\%(levelname)s}\StringTok{ {-} }\SpecialCharTok{\%(message)s}\StringTok{\textquotesingle{}}\NormalTok{)}

\CommentTok{\# {-}{-}{-}{-}{-}{-}{-}{-}{-}{-}{-}{-}{-}{-}{-}{-}{-}{-}{-}{-}{-}{-}{-}{-}{-}{-}{-}{-}{-}{-}{-}{-}{-}{-}{-}{-}{-}{-}{-}{-}{-}{-}{-}{-}{-}{-}{-}{-}{-}{-}{-}{-}{-}{-}{-}{-}{-}{-}{-}{-}}
\CommentTok{\# 读取环境变量(env.yml)以及相关配置}
\CommentTok{\# {-}{-}{-}{-}{-}{-}{-}{-}{-}{-}{-}{-}{-}{-}{-}{-}{-}{-}{-}{-}{-}{-}{-}{-}{-}{-}{-}{-}{-}{-}{-}{-}{-}{-}{-}{-}{-}{-}{-}{-}{-}{-}{-}{-}{-}{-}{-}{-}{-}{-}{-}{-}{-}{-}{-}{-}{-}{-}{-}{-}}
\KeywordTok{def}\NormalTok{ read\_env(env\_file}\OperatorTok{=}\StringTok{"env.yml"}\NormalTok{):}
    \ControlFlowTok{try}\NormalTok{:}
        \ControlFlowTok{with} \BuiltInTok{open}\NormalTok{(env\_file, }\StringTok{"r"}\NormalTok{, encoding}\OperatorTok{=}\StringTok{"utf{-}8"}\NormalTok{) }\ImportTok{as}\NormalTok{ f:}
\NormalTok{            env\_config }\OperatorTok{=}\NormalTok{ yaml.safe\_load(f)}
            \ControlFlowTok{return}\NormalTok{ env\_config}
    \ControlFlowTok{except} \PreprocessorTok{Exception} \ImportTok{as}\NormalTok{ e:}
\NormalTok{        logging.error(}\SpecialStringTok{f"读取配置文件 }\SpecialCharTok{\{}\NormalTok{env\_file}\SpecialCharTok{\}}\SpecialStringTok{ 失败: }\SpecialCharTok{\{}\NormalTok{e}\SpecialCharTok{\}}\SpecialStringTok{"}\NormalTok{)}
        \ControlFlowTok{return}\NormalTok{ \{\}}

\NormalTok{env\_config }\OperatorTok{=}\NormalTok{ read\_env()}
\CommentTok{\# 从环境中获取 API Key 与 Zotero 的用户信息}
\NormalTok{BAI\_LIAN\_API\_KEY }\OperatorTok{=}\NormalTok{ env\_config.get(}\StringTok{"Bai\_Lian\_API\_KEY"}\NormalTok{, }\StringTok{""}\NormalTok{)}
\NormalTok{openai.api\_key }\OperatorTok{=}\NormalTok{ BAI\_LIAN\_API\_KEY}
\NormalTok{openai.api\_base }\OperatorTok{=} \StringTok{"https://dashscope.aliyuncs.com/compatible{-}mode/v1"}

\NormalTok{ZOTERO\_USER\_ID }\OperatorTok{=}\NormalTok{ env\_config.get(}\StringTok{"Zotero\_user\_id"}\NormalTok{, }\StringTok{""}\NormalTok{)}
\NormalTok{ZOTERO\_API\_KEY }\OperatorTok{=}\NormalTok{ env\_config.get(}\StringTok{"Zotero\_API\_KEY"}\NormalTok{, }\StringTok{""}\NormalTok{)}
\NormalTok{ZOTERO\_UPLOAD\_URL }\OperatorTok{=} \SpecialStringTok{f"https://api.zotero.org/users/}\SpecialCharTok{\{}\NormalTok{ZOTERO\_USER\_ID}\SpecialCharTok{\}}\SpecialStringTok{/items"}

\CommentTok{\# {-}{-}{-}{-}{-}{-}{-}{-}{-}{-}{-}{-}{-}{-}{-}{-}{-}{-}{-}{-}{-}{-}{-}{-}{-}{-}{-}{-}{-}{-}{-}{-}{-}{-}{-}{-}{-}{-}{-}{-}{-}{-}{-}{-}{-}{-}{-}{-}{-}{-}{-}{-}{-}{-}{-}{-}{-}{-}{-}{-}}
\CommentTok{\# 读取配置文件(config.yaml)}
\CommentTok{\# {-}{-}{-}{-}{-}{-}{-}{-}{-}{-}{-}{-}{-}{-}{-}{-}{-}{-}{-}{-}{-}{-}{-}{-}{-}{-}{-}{-}{-}{-}{-}{-}{-}{-}{-}{-}{-}{-}{-}{-}{-}{-}{-}{-}{-}{-}{-}{-}{-}{-}{-}{-}{-}{-}{-}{-}{-}{-}{-}{-}}
\KeywordTok{def}\NormalTok{ read\_config(config\_file}\OperatorTok{=}\StringTok{"config.yaml"}\NormalTok{):}
    \ControlFlowTok{try}\NormalTok{:}
        \ControlFlowTok{with} \BuiltInTok{open}\NormalTok{(config\_file, }\StringTok{"r"}\NormalTok{, encoding}\OperatorTok{=}\StringTok{"utf{-}8"}\NormalTok{) }\ImportTok{as}\NormalTok{ f:}
\NormalTok{            config\_data }\OperatorTok{=}\NormalTok{ yaml.safe\_load(f)}
            \ControlFlowTok{return}\NormalTok{ config\_data}
    \ControlFlowTok{except} \PreprocessorTok{Exception} \ImportTok{as}\NormalTok{ e:}
\NormalTok{        logging.error(}\SpecialStringTok{f"读取配置文件 }\SpecialCharTok{\{}\NormalTok{config\_file}\SpecialCharTok{\}}\SpecialStringTok{ 时出错: }\SpecialCharTok{\{}\NormalTok{e}\SpecialCharTok{\}}\SpecialStringTok{"}\NormalTok{)}
        \ControlFlowTok{return}\NormalTok{ \{\}}

\CommentTok{\# {-}{-}{-}{-}{-}{-}{-}{-}{-}{-}{-}{-}{-}{-}{-}{-}{-}{-}{-}{-}{-}{-}{-}{-}{-}{-}{-}{-}{-}{-}{-}{-}{-}{-}{-}{-}{-}{-}{-}{-}{-}{-}{-}{-}{-}{-}{-}{-}{-}{-}{-}{-}{-}{-}{-}{-}{-}{-}{-}{-}}
\CommentTok{\# 函数:fetch\_crossref}
\CommentTok{\# 作用:调用 CrossRef API 根据查询条件获取文献数据}
\CommentTok{\# {-}{-}{-}{-}{-}{-}{-}{-}{-}{-}{-}{-}{-}{-}{-}{-}{-}{-}{-}{-}{-}{-}{-}{-}{-}{-}{-}{-}{-}{-}{-}{-}{-}{-}{-}{-}{-}{-}{-}{-}{-}{-}{-}{-}{-}{-}{-}{-}{-}{-}{-}{-}{-}{-}{-}{-}{-}{-}{-}{-}}
\KeywordTok{def}\NormalTok{ fetch\_crossref(query, year\_range):}
\NormalTok{    url }\OperatorTok{=} \StringTok{"https://api.crossref.org/works"}
    \ControlFlowTok{try}\NormalTok{:}
\NormalTok{        year\_from, year\_until }\OperatorTok{=}\NormalTok{ year\_range.split(}\StringTok{\textquotesingle{}{-}\textquotesingle{}}\NormalTok{)}
    \ControlFlowTok{except} \PreprocessorTok{ValueError}\NormalTok{:}
\NormalTok{        logging.error(}\StringTok{"出版年份范围格式错误,应为 \textquotesingle{}YYYY{-}YYYY\textquotesingle{}"}\NormalTok{)}
        \ControlFlowTok{return}\NormalTok{ []}

\NormalTok{    params }\OperatorTok{=}\NormalTok{ \{}
        \StringTok{"query"}\NormalTok{: query,}
        \StringTok{"filter"}\NormalTok{: }\SpecialStringTok{f"from{-}pub{-}date:}\SpecialCharTok{\{}\NormalTok{year\_from}\SpecialCharTok{\}}\SpecialStringTok{,until{-}pub{-}date:}\SpecialCharTok{\{}\NormalTok{year\_until}\SpecialCharTok{\}}\SpecialStringTok{"}\NormalTok{,}
        \StringTok{"rows"}\NormalTok{: }\DecValTok{50}  \CommentTok{\# 可根据需要调整获取文献数量}
\NormalTok{    \}}
\NormalTok{    headers }\OperatorTok{=}\NormalTok{ \{}
        \StringTok{"User{-}Agent"}\NormalTok{: }\StringTok{"DocumentManager/1.0 (mailto:your\_email@example.com)"}  \CommentTok{\# 请替换为你自己的邮箱}
\NormalTok{    \}}
    \ControlFlowTok{try}\NormalTok{:}
\NormalTok{        response }\OperatorTok{=}\NormalTok{ requests.get(url, params}\OperatorTok{=}\NormalTok{params, headers}\OperatorTok{=}\NormalTok{headers)}
        \ControlFlowTok{if}\NormalTok{ response.status\_code }\OperatorTok{==} \DecValTok{200}\NormalTok{:}
\NormalTok{            items }\OperatorTok{=}\NormalTok{ response.json().get(}\StringTok{\textquotesingle{}message\textquotesingle{}}\NormalTok{, \{\}).get(}\StringTok{\textquotesingle{}items\textquotesingle{}}\NormalTok{, [])}
            \ControlFlowTok{return}\NormalTok{ items}
        \ControlFlowTok{else}\NormalTok{:}
\NormalTok{            logging.error(}\SpecialStringTok{f"CrossRef请求失败,状态码:}\SpecialCharTok{\{}\NormalTok{response}\SpecialCharTok{.}\NormalTok{status\_code}\SpecialCharTok{\}}\SpecialStringTok{"}\NormalTok{)}
            \ControlFlowTok{return}\NormalTok{ []}
    \ControlFlowTok{except} \PreprocessorTok{Exception} \ImportTok{as}\NormalTok{ e:}
\NormalTok{        logging.error(}\SpecialStringTok{f"请求 CrossRef 时出错: }\SpecialCharTok{\{}\NormalTok{e}\SpecialCharTok{\}}\SpecialStringTok{"}\NormalTok{)}
        \ControlFlowTok{return}\NormalTok{ []}
\end{Highlighting}
\end{Shaded}

\section{筛选文献}\label{ux7b5bux9009ux6587ux732e}

\begin{Shaded}
\begin{Highlighting}[]
\CommentTok{\# {-}{-}{-}{-}{-}{-}{-}{-}{-}{-}{-}{-}{-}{-}{-}{-}{-}{-}{-}{-}{-}{-}{-}{-}{-}{-}{-}{-}{-}{-}{-}{-}{-}{-}{-}{-}{-}{-}{-}{-}{-}{-}{-}{-}{-}{-}{-}{-}{-}{-}{-}{-}{-}{-}{-}{-}{-}{-}{-}{-}}
\CommentTok{\# 函数:qwen\_api\_call}
\CommentTok{\# 作用:调用 Qwen API 生成对话回复(主要用于获取文献相关性评分)}
\CommentTok{\# {-}{-}{-}{-}{-}{-}{-}{-}{-}{-}{-}{-}{-}{-}{-}{-}{-}{-}{-}{-}{-}{-}{-}{-}{-}{-}{-}{-}{-}{-}{-}{-}{-}{-}{-}{-}{-}{-}{-}{-}{-}{-}{-}{-}{-}{-}{-}{-}{-}{-}{-}{-}{-}{-}{-}{-}{-}{-}{-}{-}}
\KeywordTok{def}\NormalTok{ qwen\_api\_call(prompt, model}\OperatorTok{=}\StringTok{"qwen{-}max"}\NormalTok{):}
    \ControlFlowTok{try}\NormalTok{:}
\NormalTok{        completion }\OperatorTok{=}\NormalTok{ openai.ChatCompletion.create(}
\NormalTok{            model}\OperatorTok{=}\NormalTok{model,}
\NormalTok{            messages}\OperatorTok{=}\NormalTok{[}
\NormalTok{                \{}\StringTok{"role"}\NormalTok{: }\StringTok{"system"}\NormalTok{, }\StringTok{"content"}\NormalTok{: }\StringTok{"You are a helpful assistant."}\NormalTok{\},}
\NormalTok{                \{}\StringTok{"role"}\NormalTok{: }\StringTok{"user"}\NormalTok{, }\StringTok{"content"}\NormalTok{: prompt\}}
\NormalTok{            ],}
\NormalTok{            temperature}\OperatorTok{=}\FloatTok{0.0}\NormalTok{,  }\CommentTok{\# 保证输出稳定}
\NormalTok{            top\_p}\OperatorTok{=}\FloatTok{1.0}\NormalTok{,}
\NormalTok{            max\_tokens}\OperatorTok{=}\DecValTok{10}    \CommentTok{\# 输出长度,可根据需要调整}
\NormalTok{        )}
        \ControlFlowTok{return}\NormalTok{ completion.choices[}\DecValTok{0}\NormalTok{].message.content.strip()}
    \ControlFlowTok{except} \PreprocessorTok{Exception} \ImportTok{as}\NormalTok{ e:}
\NormalTok{        logging.error(}\SpecialStringTok{f"Qwen API 调用失败:}\SpecialCharTok{\{}\NormalTok{e}\SpecialCharTok{\}}\SpecialStringTok{"}\NormalTok{)}
        \ControlFlowTok{return} \StringTok{""}

\CommentTok{\# {-}{-}{-}{-}{-}{-}{-}{-}{-}{-}{-}{-}{-}{-}{-}{-}{-}{-}{-}{-}{-}{-}{-}{-}{-}{-}{-}{-}{-}{-}{-}{-}{-}{-}{-}{-}{-}{-}{-}{-}{-}{-}{-}{-}{-}{-}{-}{-}{-}{-}{-}{-}{-}{-}{-}{-}{-}{-}{-}{-}}
\CommentTok{\# 函数:ai\_filter\_qwen}
\CommentTok{\# 作用:利用 Qwen API 对文献与给定方向的相关性进行评分}
\CommentTok{\# {-}{-}{-}{-}{-}{-}{-}{-}{-}{-}{-}{-}{-}{-}{-}{-}{-}{-}{-}{-}{-}{-}{-}{-}{-}{-}{-}{-}{-}{-}{-}{-}{-}{-}{-}{-}{-}{-}{-}{-}{-}{-}{-}{-}{-}{-}{-}{-}{-}{-}{-}{-}{-}{-}{-}{-}{-}{-}{-}{-}}
\KeywordTok{def}\NormalTok{ ai\_filter\_qwen(item, direction, threshold}\OperatorTok{=}\FloatTok{0.6}\NormalTok{):}
\NormalTok{    prompt }\OperatorTok{=}\NormalTok{ textwrap.dedent(}\SpecialStringTok{f"""}
\SpecialStringTok{        请判断以下文献是否与“}\SpecialCharTok{\{}\NormalTok{direction}\SpecialCharTok{\}}\SpecialStringTok{”密切相关,并请仅返回一个0到1之间的评分(保留两位小数)。}

\SpecialStringTok{        标题:}\SpecialCharTok{\{}\NormalTok{item}\SpecialCharTok{.}\NormalTok{get(}\StringTok{\textquotesingle{}title\textquotesingle{}}\NormalTok{, }\StringTok{\textquotesingle{}无标题\textquotesingle{}}\NormalTok{)}\SpecialCharTok{\}}
\SpecialStringTok{        摘要:}\SpecialCharTok{\{}\NormalTok{item}\SpecialCharTok{.}\NormalTok{get(}\StringTok{\textquotesingle{}abstract\textquotesingle{}}\NormalTok{, }\StringTok{\textquotesingle{}无摘要\textquotesingle{}}\NormalTok{)}\SpecialCharTok{\}}

\SpecialStringTok{        评分标准:}
\SpecialStringTok{        0.00 表示完全不相关,1.00 表示高度相关。}
\SpecialStringTok{    """}\NormalTok{)}
\NormalTok{    response }\OperatorTok{=}\NormalTok{ qwen\_api\_call(prompt)}
    \ControlFlowTok{try}\NormalTok{:}
\NormalTok{        score }\OperatorTok{=} \BuiltInTok{float}\NormalTok{(response)}
\NormalTok{        logging.info(}\SpecialStringTok{f"文献《}\SpecialCharTok{\{}\NormalTok{item[}\StringTok{\textquotesingle{}title\textquotesingle{}}\NormalTok{][:}\DecValTok{30}\NormalTok{]}\SpecialCharTok{\}}\SpecialStringTok{...》评分:}\SpecialCharTok{\{}\NormalTok{score}\SpecialCharTok{\}}\SpecialStringTok{"}\NormalTok{)}
        \ControlFlowTok{return}\NormalTok{ score }\OperatorTok{\textgreater{}=}\NormalTok{ threshold}
    \ControlFlowTok{except} \PreprocessorTok{ValueError} \ImportTok{as}\NormalTok{ e:}
\NormalTok{        logging.error(}\SpecialStringTok{f"评分解析错误:}\SpecialCharTok{\{}\NormalTok{e}\SpecialCharTok{\}}\SpecialStringTok{,返回内容:}\SpecialCharTok{\{}\NormalTok{response}\SpecialCharTok{\}}\SpecialStringTok{"}\NormalTok{)}
        \ControlFlowTok{return} \VariableTok{False}
\end{Highlighting}
\end{Shaded}

\section{归档文献}\label{ux5f52ux6863ux6587ux732e}

\begin{Shaded}
\begin{Highlighting}[]

\CommentTok{\# {-}{-}{-}{-}{-}{-}{-}{-}{-}{-}{-}{-}{-}{-}{-}{-}{-}{-}{-}{-}{-}{-}{-}{-}{-}{-}{-}{-}{-}{-}{-}{-}{-}{-}{-}{-}{-}{-}{-}{-}{-}{-}{-}{-}{-}{-}{-}{-}{-}{-}{-}{-}{-}{-}{-}{-}{-}{-}{-}{-}}
\CommentTok{\# 函数:save\_filtered\_items}
\CommentTok{\# 作用:将过滤后的文献信息保存到 YAML 文件}
\CommentTok{\# {-}{-}{-}{-}{-}{-}{-}{-}{-}{-}{-}{-}{-}{-}{-}{-}{-}{-}{-}{-}{-}{-}{-}{-}{-}{-}{-}{-}{-}{-}{-}{-}{-}{-}{-}{-}{-}{-}{-}{-}{-}{-}{-}{-}{-}{-}{-}{-}{-}{-}{-}{-}{-}{-}{-}{-}{-}{-}{-}{-}}
\KeywordTok{def}\NormalTok{ save\_filtered\_items(filtered\_items, filename}\OperatorTok{=}\StringTok{"filtered\_items.yaml"}\NormalTok{):}
    \ControlFlowTok{try}\NormalTok{:}
        \ControlFlowTok{with} \BuiltInTok{open}\NormalTok{(filename, }\StringTok{"w"}\NormalTok{, encoding}\OperatorTok{=}\StringTok{"utf{-}8"}\NormalTok{) }\ImportTok{as}\NormalTok{ f:}
\NormalTok{            yaml.safe\_dump(filtered\_items, f, allow\_unicode}\OperatorTok{=}\VariableTok{True}\NormalTok{)}
\NormalTok{        logging.info(}\SpecialStringTok{f"过滤后的文献信息已保存至 }\SpecialCharTok{\{}\NormalTok{filename}\SpecialCharTok{\}}\SpecialStringTok{"}\NormalTok{)}
    \ControlFlowTok{except} \PreprocessorTok{Exception} \ImportTok{as}\NormalTok{ e:}
\NormalTok{        logging.error(}\SpecialStringTok{f"保存文件 }\SpecialCharTok{\{}\NormalTok{filename}\SpecialCharTok{\}}\SpecialStringTok{ 时出错: }\SpecialCharTok{\{}\NormalTok{e}\SpecialCharTok{\}}\SpecialStringTok{"}\NormalTok{)}

\CommentTok{\# {-}{-}{-}{-}{-}{-}{-}{-}{-}{-}{-}{-}{-}{-}{-}{-}{-}{-}{-}{-}{-}{-}{-}{-}{-}{-}{-}{-}{-}{-}{-}{-}{-}{-}{-}{-}{-}{-}{-}{-}{-}{-}{-}{-}{-}{-}{-}{-}{-}{-}{-}{-}{-}{-}{-}{-}{-}{-}{-}{-}}
\CommentTok{\# 函数:load\_filtered\_items}
\CommentTok{\# 作用:从 YAML 文件中读取之前筛选好的文献信息}
\CommentTok{\# {-}{-}{-}{-}{-}{-}{-}{-}{-}{-}{-}{-}{-}{-}{-}{-}{-}{-}{-}{-}{-}{-}{-}{-}{-}{-}{-}{-}{-}{-}{-}{-}{-}{-}{-}{-}{-}{-}{-}{-}{-}{-}{-}{-}{-}{-}{-}{-}{-}{-}{-}{-}{-}{-}{-}{-}{-}{-}{-}{-}}
\KeywordTok{def}\NormalTok{ load\_filtered\_items(filename}\OperatorTok{=}\StringTok{"filtered\_items.yaml"}\NormalTok{):}
    \ControlFlowTok{try}\NormalTok{:}
        \ControlFlowTok{with} \BuiltInTok{open}\NormalTok{(filename, }\StringTok{"r"}\NormalTok{, encoding}\OperatorTok{=}\StringTok{"utf{-}8"}\NormalTok{) }\ImportTok{as}\NormalTok{ f:}
\NormalTok{            items }\OperatorTok{=}\NormalTok{ yaml.safe\_load(f)}
        \ControlFlowTok{if}\NormalTok{ items:}
\NormalTok{            logging.info(}\SpecialStringTok{f"成功读取 }\SpecialCharTok{\{}\BuiltInTok{len}\NormalTok{(items)}\SpecialCharTok{\}}\SpecialStringTok{ 篇筛选后的文献。"}\NormalTok{)}
            \ControlFlowTok{return}\NormalTok{ items}
        \ControlFlowTok{else}\NormalTok{:}
\NormalTok{            logging.info(}\StringTok{"没有找到任何筛选后的文献。"}\NormalTok{)}
            \ControlFlowTok{return}\NormalTok{ []}
    \ControlFlowTok{except} \PreprocessorTok{Exception} \ImportTok{as}\NormalTok{ e:}
\NormalTok{        logging.error(}\SpecialStringTok{f"读取文件 }\SpecialCharTok{\{}\NormalTok{filename}\SpecialCharTok{\}}\SpecialStringTok{ 时出错: }\SpecialCharTok{\{}\NormalTok{e}\SpecialCharTok{\}}\SpecialStringTok{"}\NormalTok{)}
        \ControlFlowTok{return}\NormalTok{ []}

\CommentTok{\# {-}{-}{-}{-}{-}{-}{-}{-}{-}{-}{-}{-}{-}{-}{-}{-}{-}{-}{-}{-}{-}{-}{-}{-}{-}{-}{-}{-}{-}{-}{-}{-}{-}{-}{-}{-}{-}{-}{-}{-}{-}{-}{-}{-}{-}{-}{-}{-}{-}{-}{-}{-}{-}{-}{-}{-}{-}{-}{-}{-}}
\CommentTok{\# 函数:convert\_to\_zotero\_format}
\CommentTok{\# 作用:转换单个文献信息为符合 Zotero API 要求的数据格式}
\CommentTok{\# {-}{-}{-}{-}{-}{-}{-}{-}{-}{-}{-}{-}{-}{-}{-}{-}{-}{-}{-}{-}{-}{-}{-}{-}{-}{-}{-}{-}{-}{-}{-}{-}{-}{-}{-}{-}{-}{-}{-}{-}{-}{-}{-}{-}{-}{-}{-}{-}{-}{-}{-}{-}{-}{-}{-}{-}{-}{-}{-}{-}}
\KeywordTok{def}\NormalTok{ convert\_to\_zotero\_format(item):}
\NormalTok{    zotero\_item }\OperatorTok{=}\NormalTok{ \{}
        \StringTok{"itemType"}\NormalTok{: }\StringTok{"journalArticle"}\NormalTok{,}
        \StringTok{"title"}\NormalTok{: item.get(}\StringTok{"title"}\NormalTok{, }\StringTok{"无标题"}\NormalTok{),}
        \StringTok{"abstractNote"}\NormalTok{: item.get(}\StringTok{"abstract"}\NormalTok{, }\StringTok{""}\NormalTok{),}
        \StringTok{"creators"}\NormalTok{: [],}
        \StringTok{"date"}\NormalTok{: }\BuiltInTok{str}\NormalTok{(item.get(}\StringTok{"date"}\NormalTok{, }\StringTok{""}\NormalTok{)),}
        \StringTok{"DOI"}\NormalTok{: item.get(}\StringTok{"DOI"}\NormalTok{, }\StringTok{""}\NormalTok{)}
\NormalTok{    \}}
    \ControlFlowTok{for}\NormalTok{ author }\KeywordTok{in}\NormalTok{ item.get(}\StringTok{"authors"}\NormalTok{, []):}
        \ControlFlowTok{if} \BuiltInTok{isinstance}\NormalTok{(author, }\BuiltInTok{dict}\NormalTok{):}
            \ControlFlowTok{if} \StringTok{"firstName"} \KeywordTok{in}\NormalTok{ author }\KeywordTok{and} \StringTok{"lastName"} \KeywordTok{in}\NormalTok{ author:}
\NormalTok{                creator }\OperatorTok{=}\NormalTok{ \{}
                    \StringTok{"creatorType"}\NormalTok{: }\StringTok{"author"}\NormalTok{,}
                    \StringTok{"firstName"}\NormalTok{: author[}\StringTok{"firstName"}\NormalTok{],}
                    \StringTok{"lastName"}\NormalTok{: author[}\StringTok{"lastName"}\NormalTok{]}
\NormalTok{                \}}
            \ControlFlowTok{else}\NormalTok{:}
\NormalTok{                creator }\OperatorTok{=}\NormalTok{ \{}\StringTok{"creatorType"}\NormalTok{: }\StringTok{"author"}\NormalTok{, }\StringTok{"lastName"}\NormalTok{: author.get(}\StringTok{"name"}\NormalTok{, }\StringTok{""}\NormalTok{).strip()\}}
\NormalTok{            zotero\_item[}\StringTok{"creators"}\NormalTok{].append(creator)}
        \ControlFlowTok{elif} \BuiltInTok{isinstance}\NormalTok{(author, }\BuiltInTok{str}\NormalTok{):}
            \CommentTok{\# 如果作者仅为字符串形式}
\NormalTok{            zotero\_item[}\StringTok{"creators"}\NormalTok{].append(\{}\StringTok{"creatorType"}\NormalTok{: }\StringTok{"author"}\NormalTok{, }\StringTok{"lastName"}\NormalTok{: author.strip()\})}
    \ControlFlowTok{return}\NormalTok{ zotero\_item}

\CommentTok{\# {-}{-}{-}{-}{-}{-}{-}{-}{-}{-}{-}{-}{-}{-}{-}{-}{-}{-}{-}{-}{-}{-}{-}{-}{-}{-}{-}{-}{-}{-}{-}{-}{-}{-}{-}{-}{-}{-}{-}{-}{-}{-}{-}{-}{-}{-}{-}{-}{-}{-}{-}{-}{-}{-}{-}{-}{-}{-}{-}{-}}
\CommentTok{\# 函数:upload\_items\_to\_zotero}
\CommentTok{\# 作用:将转换后的文献数据上传到 Zotero}
\CommentTok{\# {-}{-}{-}{-}{-}{-}{-}{-}{-}{-}{-}{-}{-}{-}{-}{-}{-}{-}{-}{-}{-}{-}{-}{-}{-}{-}{-}{-}{-}{-}{-}{-}{-}{-}{-}{-}{-}{-}{-}{-}{-}{-}{-}{-}{-}{-}{-}{-}{-}{-}{-}{-}{-}{-}{-}{-}{-}{-}{-}{-}}
\KeywordTok{def}\NormalTok{ upload\_items\_to\_zotero(items):}
\NormalTok{    headers }\OperatorTok{=}\NormalTok{ \{}
        \StringTok{"Zotero{-}API{-}Key"}\NormalTok{: ZOTERO\_API\_KEY,}
        \StringTok{"Content{-}Type"}\NormalTok{: }\StringTok{"application/json"}\NormalTok{,}
        \StringTok{"Accept"}\NormalTok{: }\StringTok{"application/json"}
\NormalTok{    \}}
\NormalTok{    zotero\_items }\OperatorTok{=}\NormalTok{ [convert\_to\_zotero\_format(item) }\ControlFlowTok{for}\NormalTok{ item }\KeywordTok{in}\NormalTok{ items]}
\NormalTok{    payload }\OperatorTok{=}\NormalTok{ json.dumps(zotero\_items, ensure\_ascii}\OperatorTok{=}\VariableTok{False}\NormalTok{)}
    \ControlFlowTok{try}\NormalTok{:}
\NormalTok{        response }\OperatorTok{=}\NormalTok{ requests.post(ZOTERO\_UPLOAD\_URL, headers}\OperatorTok{=}\NormalTok{headers, data}\OperatorTok{=}\NormalTok{payload)}
        \ControlFlowTok{if}\NormalTok{ response.status\_code }\KeywordTok{in}\NormalTok{ (}\DecValTok{200}\NormalTok{, }\DecValTok{201}\NormalTok{):}
\NormalTok{            logging.info(}\StringTok{"文献上传成功!"}\NormalTok{)}
            \ControlFlowTok{try}\NormalTok{:}
\NormalTok{                resp\_json }\OperatorTok{=}\NormalTok{ response.json()}
\NormalTok{                logging.info(}\StringTok{"服务器返回的信息:"}\NormalTok{)}
\NormalTok{                logging.info(json.dumps(resp\_json, ensure\_ascii}\OperatorTok{=}\VariableTok{False}\NormalTok{, indent}\OperatorTok{=}\DecValTok{2}\NormalTok{))}
            \ControlFlowTok{except} \PreprocessorTok{Exception} \ImportTok{as}\NormalTok{ err:}
\NormalTok{                logging.error(}\SpecialStringTok{f"解析服务器返回数据时出错:}\SpecialCharTok{\{}\NormalTok{err}\SpecialCharTok{\}}\SpecialStringTok{"}\NormalTok{)}
        \ControlFlowTok{else}\NormalTok{:}
\NormalTok{            logging.error(}\SpecialStringTok{f"上传失败,状态码: }\SpecialCharTok{\{}\NormalTok{response}\SpecialCharTok{.}\NormalTok{status\_code}\SpecialCharTok{\}}\SpecialStringTok{"}\NormalTok{)}
\NormalTok{            logging.error(}\SpecialStringTok{f"响应内容:}\SpecialCharTok{\{}\NormalTok{response}\SpecialCharTok{.}\NormalTok{text}\SpecialCharTok{\}}\SpecialStringTok{"}\NormalTok{)}
    \ControlFlowTok{except} \PreprocessorTok{Exception} \ImportTok{as}\NormalTok{ e:}
\NormalTok{        logging.error(}\SpecialStringTok{f"上传过程中出现错误:}\SpecialCharTok{\{}\NormalTok{e}\SpecialCharTok{\}}\SpecialStringTok{"}\NormalTok{)}
\end{Highlighting}
\end{Shaded}

\begin{Shaded}
\begin{Highlighting}[]
\CommentTok{\# {-}{-}{-}{-}{-}{-}{-}{-}{-}{-}{-}{-}{-}{-}{-}{-}{-}{-}{-}{-}{-}{-}{-}{-}{-}{-}{-}{-}{-}{-}{-}{-}{-}{-}{-}{-}{-}{-}{-}{-}{-}{-}{-}{-}{-}{-}{-}{-}{-}{-}{-}{-}{-}{-}{-}{-}{-}{-}{-}{-}}
\CommentTok{\# 主函数:整合跨平台文献筛选与上传流程}
\CommentTok{\# {-}{-}{-}{-}{-}{-}{-}{-}{-}{-}{-}{-}{-}{-}{-}{-}{-}{-}{-}{-}{-}{-}{-}{-}{-}{-}{-}{-}{-}{-}{-}{-}{-}{-}{-}{-}{-}{-}{-}{-}{-}{-}{-}{-}{-}{-}{-}{-}{-}{-}{-}{-}{-}{-}{-}{-}{-}{-}{-}{-}}
\KeywordTok{def}\NormalTok{ main():}
    \CommentTok{\# 第一阶段:根据配置文件,从 CrossRef 获取文献并使用 Qwen API 过滤}
\NormalTok{    config }\OperatorTok{=}\NormalTok{ read\_config()}
\NormalTok{    direction }\OperatorTok{=}\NormalTok{ config.get(}\StringTok{"direction"}\NormalTok{, }\StringTok{""}\NormalTok{)}
\NormalTok{    keywords }\OperatorTok{=} \StringTok{" "}\NormalTok{.join(config.get(}\StringTok{"keywords"}\NormalTok{, []))}
\NormalTok{    publication\_year }\OperatorTok{=}\NormalTok{ config.get(}\StringTok{"publication\_year"}\NormalTok{, }\StringTok{""}\NormalTok{)}
\NormalTok{    query }\OperatorTok{=} \SpecialStringTok{f"}\SpecialCharTok{\{}\NormalTok{direction}\SpecialCharTok{\}}\SpecialStringTok{ }\SpecialCharTok{\{}\NormalTok{keywords}\SpecialCharTok{\}}\SpecialStringTok{"}\NormalTok{.strip()}
    \ControlFlowTok{if} \KeywordTok{not}\NormalTok{ (direction }\KeywordTok{and}\NormalTok{ publication\_year):}
\NormalTok{        logging.error(}\StringTok{"配置文件缺少 \textquotesingle{}direction\textquotesingle{} 或 \textquotesingle{}publication\_year\textquotesingle{} 参数,请检查 config.yaml。"}\NormalTok{)}
        \ControlFlowTok{return}

\NormalTok{    items }\OperatorTok{=}\NormalTok{ fetch\_crossref(query, publication\_year)}
    \ControlFlowTok{if} \KeywordTok{not}\NormalTok{ items:}
\NormalTok{        logging.info(}\StringTok{"未获取到任何文献。"}\NormalTok{)}
        \ControlFlowTok{return}

\NormalTok{    processed\_items }\OperatorTok{=}\NormalTok{ []}
    \ControlFlowTok{for}\NormalTok{ item }\KeywordTok{in}\NormalTok{ items:}
\NormalTok{        processed }\OperatorTok{=}\NormalTok{ \{}
            \StringTok{"title"}\NormalTok{: item.get(}\StringTok{"title"}\NormalTok{, [}\StringTok{"无标题"}\NormalTok{])[}\DecValTok{0}\NormalTok{] }\ControlFlowTok{if}\NormalTok{ item.get(}\StringTok{"title"}\NormalTok{) }\ControlFlowTok{else} \StringTok{"无标题"}\NormalTok{,}
            \StringTok{"abstract"}\NormalTok{: item.get(}\StringTok{"abstract"}\NormalTok{, }\StringTok{"无摘要"}\NormalTok{),}
            \StringTok{"authors"}\NormalTok{: item.get(}\StringTok{"author"}\NormalTok{, []),}
            \StringTok{"date"}\NormalTok{: item.get(}\StringTok{"issued"}\NormalTok{, \{\}).get(}\StringTok{"date{-}parts"}\NormalTok{, [[}\StringTok{""}\NormalTok{]])[}\DecValTok{0}\NormalTok{][}\DecValTok{0}\NormalTok{],}
            \StringTok{"DOI"}\NormalTok{: item.get(}\StringTok{"DOI"}\NormalTok{, }\StringTok{""}\NormalTok{)}
\NormalTok{        \}}
\NormalTok{        processed\_items.append(processed)}
\NormalTok{    logging.info(}\SpecialStringTok{f"初步获取到 }\SpecialCharTok{\{}\BuiltInTok{len}\NormalTok{(processed\_items)}\SpecialCharTok{\}}\SpecialStringTok{ 篇文献。"}\NormalTok{)}

\NormalTok{    filtered\_items }\OperatorTok{=}\NormalTok{ []}
    \ControlFlowTok{for}\NormalTok{ item }\KeywordTok{in}\NormalTok{ processed\_items:}
        \ControlFlowTok{if}\NormalTok{ ai\_filter\_qwen(item, direction):}
\NormalTok{            filtered\_items.append(item)}
\NormalTok{            logging.info(}\SpecialStringTok{f"文献《}\SpecialCharTok{\{}\NormalTok{item[}\StringTok{\textquotesingle{}title\textquotesingle{}}\NormalTok{]}\SpecialCharTok{\}}\SpecialStringTok{》通过 Qwen 过滤。"}\NormalTok{)}
        \ControlFlowTok{else}\NormalTok{:}
\NormalTok{            logging.info(}\SpecialStringTok{f"文献《}\SpecialCharTok{\{}\NormalTok{item[}\StringTok{\textquotesingle{}title\textquotesingle{}}\NormalTok{]}\SpecialCharTok{\}}\SpecialStringTok{》未通过 Qwen 过滤。"}\NormalTok{)}
\NormalTok{        time.sleep(}\DecValTok{1}\NormalTok{)  }\CommentTok{\# 暂停1秒,避免请求过于频繁}

\NormalTok{    logging.info(}\SpecialStringTok{f"经过 Qwen 过滤后,剩余文献数量:}\SpecialCharTok{\{}\BuiltInTok{len}\NormalTok{(filtered\_items)}\SpecialCharTok{\}}\SpecialStringTok{"}\NormalTok{)}
\NormalTok{    save\_filtered\_items(filtered\_items, filename}\OperatorTok{=}\StringTok{"filtered\_items.yaml"}\NormalTok{)}

    \CommentTok{\# 第二阶段:从保存结果中读取文献数据并上传至 Zotero}
\NormalTok{    items\_to\_upload }\OperatorTok{=}\NormalTok{ load\_filtered\_items(}\StringTok{"filtered\_items.yaml"}\NormalTok{)}
    \ControlFlowTok{if}\NormalTok{ items\_to\_upload:}
\NormalTok{        logging.info(}\StringTok{"开始上传文献到 Zotero ..."}\NormalTok{)}
\NormalTok{        upload\_items\_to\_zotero(items\_to\_upload)}
    \ControlFlowTok{else}\NormalTok{:}
\NormalTok{        logging.info(}\StringTok{"没有可上传的文献。"}\NormalTok{)}

\ControlFlowTok{if} \VariableTok{\_\_name\_\_} \OperatorTok{==} \StringTok{"\_\_main\_\_"}\NormalTok{:}
\NormalTok{    main()}
\end{Highlighting}
\end{Shaded}

上面的代码主要完成了以下功能:

\begin{itemize}
\tightlist
\item
  从 CrossRef API 获取文献数据
\item
  使用 Qwen API 对文献进行相关性评分
\item
  将筛选后的文献保存到 YAML 文件
\item
  将筛选后的文献转换为 Zotero API 所需格式
\item
  将文献上传到 Zotero
\end{itemize}

\section{文库更新}\label{ux6587ux5e93ux66f4ux65b0}

\begin{Shaded}
\begin{Highlighting}[]
\ImportTok{import}\NormalTok{ requests}
\ImportTok{import}\NormalTok{ json}
\ImportTok{import}\NormalTok{ os}
\ImportTok{import}\NormalTok{ yaml}

\CommentTok{\# 读取配置文件}
\KeywordTok{def}\NormalTok{ load\_config():}
    \ControlFlowTok{with} \BuiltInTok{open}\NormalTok{(}\StringTok{\textquotesingle{}env.yml\textquotesingle{}}\NormalTok{, }\StringTok{\textquotesingle{}r\textquotesingle{}}\NormalTok{, encoding}\OperatorTok{=}\StringTok{\textquotesingle{}utf{-}8\textquotesingle{}}\NormalTok{) }\ImportTok{as}\NormalTok{ f:}
        \ControlFlowTok{return}\NormalTok{ yaml.safe\_load(f)}

\CommentTok{\# 从配置文件中读取参数(使用 dict.get 提供默认值)}
\NormalTok{config }\OperatorTok{=}\NormalTok{ load\_config()}
\NormalTok{ZOTERO\_USER\_ID }\OperatorTok{=}\NormalTok{ config.get(}\StringTok{\textquotesingle{}Zotero\_user\_id\textquotesingle{}}\NormalTok{)}
\NormalTok{ZOTERO\_API\_KEY }\OperatorTok{=}\NormalTok{ config.get(}\StringTok{\textquotesingle{}Zotero\_API\_KEY\textquotesingle{}}\NormalTok{)}
\NormalTok{ZOTERO\_BASE\_URL }\OperatorTok{=}\NormalTok{ config.get(}\StringTok{\textquotesingle{}Zotero\_BASE\_URL\textquotesingle{}}\NormalTok{, }\StringTok{\textquotesingle{}https://api.zotero.org\textquotesingle{}}\NormalTok{)  }\CommentTok{\# 如果不存在则使用默认值}

\NormalTok{headers }\OperatorTok{=}\NormalTok{ \{}\StringTok{"Authorization"}\NormalTok{: }\SpecialStringTok{f"Bearer }\SpecialCharTok{\{}\NormalTok{ZOTERO\_API\_KEY}\SpecialCharTok{\}}\SpecialStringTok{"}\NormalTok{\}}

\CommentTok{\# 1. 获取 Zotero 中的所有集合}
\KeywordTok{def}\NormalTok{ fetch\_collections():}
\NormalTok{    url }\OperatorTok{=} \SpecialStringTok{f"}\SpecialCharTok{\{}\NormalTok{ZOTERO\_BASE\_URL}\SpecialCharTok{\}}\SpecialStringTok{/users/}\SpecialCharTok{\{}\NormalTok{ZOTERO\_USER\_ID}\SpecialCharTok{\}}\SpecialStringTok{/collections"}
    \ControlFlowTok{try}\NormalTok{:}
\NormalTok{        response }\OperatorTok{=}\NormalTok{ requests.get(url, headers}\OperatorTok{=}\NormalTok{headers)}
\NormalTok{        response.raise\_for\_status()}
        \ControlFlowTok{return}\NormalTok{ response.json()}
    \ControlFlowTok{except} \PreprocessorTok{Exception} \ImportTok{as}\NormalTok{ e:}
        \BuiltInTok{print}\NormalTok{(}\SpecialStringTok{f"Error fetching collections: }\SpecialCharTok{\{}\NormalTok{e}\SpecialCharTok{\}}\SpecialStringTok{"}\NormalTok{)}
        \ControlFlowTok{return}\NormalTok{ []}

\CommentTok{\# 2. 获取某个集合中的 PDF 文件}
\KeywordTok{def}\NormalTok{ fetch\_pdf\_items(collection\_id):}
\NormalTok{    url }\OperatorTok{=} \SpecialStringTok{f"}\SpecialCharTok{\{}\NormalTok{ZOTERO\_BASE\_URL}\SpecialCharTok{\}}\SpecialStringTok{/users/}\SpecialCharTok{\{}\NormalTok{ZOTERO\_USER\_ID}\SpecialCharTok{\}}\SpecialStringTok{/collections/}\SpecialCharTok{\{}\NormalTok{collection\_id}\SpecialCharTok{\}}\SpecialStringTok{/items"}
\NormalTok{    params }\OperatorTok{=}\NormalTok{ \{}
        \StringTok{"format"}\NormalTok{: }\StringTok{"json"}\NormalTok{,}
        \StringTok{"include"}\NormalTok{: }\StringTok{"data"}\NormalTok{,}
        \StringTok{"limit"}\NormalTok{: }\DecValTok{100}\NormalTok{,}
\NormalTok{    \}}
    \ControlFlowTok{try}\NormalTok{:}
\NormalTok{        response }\OperatorTok{=}\NormalTok{ requests.get(url, headers}\OperatorTok{=}\NormalTok{headers, params}\OperatorTok{=}\NormalTok{params)}
\NormalTok{        response.raise\_for\_status()}
\NormalTok{        items }\OperatorTok{=}\NormalTok{ response.json()}
\NormalTok{        pdf\_items }\OperatorTok{=}\NormalTok{ [}
\NormalTok{            item }\ControlFlowTok{for}\NormalTok{ item }\KeywordTok{in}\NormalTok{ items}
            \ControlFlowTok{if}\NormalTok{ item[}\StringTok{"data"}\NormalTok{].get(}\StringTok{"itemType"}\NormalTok{) }\OperatorTok{==} \StringTok{"attachment"} \KeywordTok{and} 
\NormalTok{               item[}\StringTok{"data"}\NormalTok{].get(}\StringTok{"contentType"}\NormalTok{, }\StringTok{""}\NormalTok{).startswith(}\StringTok{"application/pdf"}\NormalTok{)}
\NormalTok{        ]}
        \ControlFlowTok{return}\NormalTok{ pdf\_items}
    \ControlFlowTok{except} \PreprocessorTok{Exception} \ImportTok{as}\NormalTok{ e:}
        \BuiltInTok{print}\NormalTok{(}\SpecialStringTok{f"Error fetching PDF items for collection }\SpecialCharTok{\{}\NormalTok{collection\_id}\SpecialCharTok{\}}\SpecialStringTok{: }\SpecialCharTok{\{}\NormalTok{e}\SpecialCharTok{\}}\SpecialStringTok{"}\NormalTok{)}
        \ControlFlowTok{return}\NormalTok{ []}

\CommentTok{\# 3. 获取当前状态(已有 PDF 文件)并返回}
\KeywordTok{def}\NormalTok{ fetch\_current\_state():}
    \ControlFlowTok{if}\NormalTok{ os.path.exists(}\StringTok{"zotero\_state.json"}\NormalTok{):}
        \ControlFlowTok{with} \BuiltInTok{open}\NormalTok{(}\StringTok{"zotero\_state.json"}\NormalTok{, }\StringTok{"r"}\NormalTok{, encoding}\OperatorTok{=}\StringTok{"utf{-}8"}\NormalTok{) }\ImportTok{as}\NormalTok{ f:}
            \ControlFlowTok{try}\NormalTok{:}
                \ControlFlowTok{return}\NormalTok{ json.load(f)}
            \ControlFlowTok{except} \PreprocessorTok{Exception} \ImportTok{as}\NormalTok{ e:}
                \BuiltInTok{print}\NormalTok{(}\StringTok{"Error reading state file, starting with an empty state:"}\NormalTok{, e)}
                \ControlFlowTok{return}\NormalTok{ \{\}}
    \ControlFlowTok{return}\NormalTok{ \{\}}

\CommentTok{\# 4. 保存当前状态并打印实时状态(只打印关键结果)}
\KeywordTok{def}\NormalTok{ save\_current\_state(state):}
    \BuiltInTok{print}\NormalTok{(}\StringTok{" 当前实时状态:"}\NormalTok{)}
    \BuiltInTok{print}\NormalTok{(json.dumps(state, ensure\_ascii}\OperatorTok{=}\VariableTok{False}\NormalTok{, indent}\OperatorTok{=}\DecValTok{4}\NormalTok{))}
    \ControlFlowTok{with} \BuiltInTok{open}\NormalTok{(}\StringTok{"zotero\_state.json"}\NormalTok{, }\StringTok{"w"}\NormalTok{, encoding}\OperatorTok{=}\StringTok{"utf{-}8"}\NormalTok{) }\ImportTok{as}\NormalTok{ f:}
\NormalTok{        json.dump(state, f, ensure\_ascii}\OperatorTok{=}\VariableTok{False}\NormalTok{, indent}\OperatorTok{=}\DecValTok{4}\NormalTok{)}
    \BuiltInTok{print}\NormalTok{(}\StringTok{" 当前状态已保存为 zotero\_state.json"}\NormalTok{)}

\CommentTok{\# 5. 对比更新:只统计并打印新增的 PDF 项}
\KeywordTok{def}\NormalTok{ check\_updates():}
\NormalTok{    collections }\OperatorTok{=}\NormalTok{ fetch\_collections()}
\NormalTok{    previous\_state }\OperatorTok{=}\NormalTok{ fetch\_current\_state()}
\NormalTok{    new\_state }\OperatorTok{=}\NormalTok{ \{\}}
\NormalTok{    update\_summary }\OperatorTok{=}\NormalTok{ \{}\StringTok{"新增"}\NormalTok{: []\}}

    \ControlFlowTok{for}\NormalTok{ collection }\KeywordTok{in}\NormalTok{ collections:}
\NormalTok{        collection\_id }\OperatorTok{=}\NormalTok{ collection[}\StringTok{"key"}\NormalTok{]}
\NormalTok{        collection\_name }\OperatorTok{=}\NormalTok{ collection[}\StringTok{"data"}\NormalTok{].get(}\StringTok{"name"}\NormalTok{, }\StringTok{"Unnamed Collection"}\NormalTok{)}
\NormalTok{        pdf\_items }\OperatorTok{=}\NormalTok{ fetch\_pdf\_items(collection\_id)}
        \ControlFlowTok{for}\NormalTok{ pdf }\KeywordTok{in}\NormalTok{ pdf\_items:}
\NormalTok{            item\_key }\OperatorTok{=}\NormalTok{ pdf[}\StringTok{"data"}\NormalTok{].get(}\StringTok{"key"}\NormalTok{)}
\NormalTok{            new\_state[item\_key] }\OperatorTok{=}\NormalTok{ \{}\StringTok{"collection"}\NormalTok{: collection\_name\}}
            \CommentTok{\# 如果该 PDF 不在之前状态中,则视为新增项}
            \ControlFlowTok{if}\NormalTok{ item\_key }\KeywordTok{not} \KeywordTok{in}\NormalTok{ previous\_state:}
\NormalTok{                update\_summary[}\StringTok{"新增"}\NormalTok{].append(\{item\_key: \{}\StringTok{"collection"}\NormalTok{: collection\_name\}\})}
                
    \BuiltInTok{print}\NormalTok{(}\StringTok{" 更新信息统计:"}\NormalTok{)}
    \BuiltInTok{print}\NormalTok{(}\SpecialStringTok{f"新增:}\SpecialCharTok{\{}\BuiltInTok{len}\NormalTok{(update\_summary[}\StringTok{\textquotesingle{}新增\textquotesingle{}}\NormalTok{])}\SpecialCharTok{\}}\SpecialStringTok{"}\NormalTok{)}
    \ControlFlowTok{if}\NormalTok{ update\_summary[}\StringTok{"新增"}\NormalTok{]:}
        \BuiltInTok{print}\NormalTok{(}\StringTok{"详细新增项目:"}\NormalTok{)}
        \ControlFlowTok{for}\NormalTok{ item }\KeywordTok{in}\NormalTok{ update\_summary[}\StringTok{"新增"}\NormalTok{]:}
            \BuiltInTok{print}\NormalTok{(item)}
    
    \CommentTok{\# 保存新的实时状态}
\NormalTok{    save\_current\_state(new\_state)}
    \ControlFlowTok{return}\NormalTok{ update\_summary}

\CommentTok{\# 主函数:运行更新检查并保存更新信息}
\ControlFlowTok{if} \VariableTok{\_\_name\_\_} \OperatorTok{==} \StringTok{"\_\_main\_\_"}\NormalTok{:}
\NormalTok{    update\_summary }\OperatorTok{=}\NormalTok{ check\_updates()}
    \ControlFlowTok{with} \BuiltInTok{open}\NormalTok{(}\StringTok{"zotero\_update\_summary.json"}\NormalTok{, }\StringTok{"w"}\NormalTok{, encoding}\OperatorTok{=}\StringTok{"utf{-}8"}\NormalTok{) }\ImportTok{as}\NormalTok{ f:}
\NormalTok{        json.dump(update\_summary, f, ensure\_ascii}\OperatorTok{=}\VariableTok{False}\NormalTok{, indent}\OperatorTok{=}\DecValTok{4}\NormalTok{)}
    \BuiltInTok{print}\NormalTok{(}\StringTok{" 更新信息已保存为 zotero\_update\_summary.json"}\NormalTok{)}
\end{Highlighting}
\end{Shaded}

\section{解读文献}\label{ux89e3ux8bfbux6587ux732e}

\begin{Shaded}
\begin{Highlighting}[]
\ImportTok{import}\NormalTok{ os}
\ImportTok{import}\NormalTok{ json}
\ImportTok{import}\NormalTok{ yaml}
\ImportTok{from}\NormalTok{ pathlib }\ImportTok{import}\NormalTok{ Path}
\ImportTok{from}\NormalTok{ openai }\ImportTok{import}\NormalTok{ OpenAI}

\CommentTok{\# ========== 步骤 1:读取配置 ==========}
\ControlFlowTok{with} \BuiltInTok{open}\NormalTok{(}\StringTok{"env.yml"}\NormalTok{, }\StringTok{"r"}\NormalTok{, encoding}\OperatorTok{=}\StringTok{"utf{-}8"}\NormalTok{) }\ImportTok{as} \BuiltInTok{file}\NormalTok{:}
\NormalTok{    env\_config }\OperatorTok{=}\NormalTok{ yaml.safe\_load(}\BuiltInTok{file}\NormalTok{)}

\NormalTok{Bai\_Lian\_API\_KEY }\OperatorTok{=}\NormalTok{ env\_config.get(}\StringTok{"Bai\_Lian\_API\_KEY"}\NormalTok{)}
\NormalTok{Zotero\_Storage }\OperatorTok{=}\NormalTok{ env\_config.get(}\StringTok{"Zotero\_Storage"}\NormalTok{)}

\ControlFlowTok{if} \KeywordTok{not}\NormalTok{ Bai\_Lian\_API\_KEY:}
    \ControlFlowTok{raise} \PreprocessorTok{ValueError}\NormalTok{(}\StringTok{"API Key 未正确读取,请检查 env.yml 文件!"}\NormalTok{)}

\ControlFlowTok{if} \KeywordTok{not}\NormalTok{ os.path.exists(Zotero\_Storage):}
    \ControlFlowTok{raise} \PreprocessorTok{FileNotFoundError}\NormalTok{(}\SpecialStringTok{f"Zotero\_Storage 路径 }\SpecialCharTok{\{}\NormalTok{Zotero\_Storage}\SpecialCharTok{\}}\SpecialStringTok{ 不存在!"}\NormalTok{)}

\CommentTok{\# ========== 步骤 2:设置客户端 ==========}
\NormalTok{client }\OperatorTok{=}\NormalTok{ OpenAI(}
\NormalTok{    api\_key}\OperatorTok{=}\NormalTok{Bai\_Lian\_API\_KEY,}
\NormalTok{    base\_url}\OperatorTok{=}\StringTok{"https://dashscope.aliyuncs.com/compatible{-}mode/v1"}\NormalTok{,}
\NormalTok{)}

\CommentTok{\# ========== 步骤 3:定义 prompt 模板 ==========}
\NormalTok{PROMPT\_TEMPLATE }\OperatorTok{=} \StringTok{"""请按照以下步骤对文献进行深入解读和分析,确保结果逻辑清晰、内容全面:}
\StringTok{\# 此处替换为文章标题}

\StringTok{\#\# 基本信息提取}

\StringTok{提取作者、发表年份、期刊名称等关键信息。}

\StringTok{\#\# 研究背景与目的}

\StringTok{总结文献的研究背景,说明研究所解决的问题或提出的假设。}
\StringTok{明确指出作者的研究目的和研究动机。}

\StringTok{\#\# 研究方法}

\StringTok{描述作者采用的研究方法(如实验、调查、建模、定量/定性分析等)。}
\StringTok{解释数据来源、采集方式以及实验设计或分析框架。}
\StringTok{分析所选方法的优势和可能存在的局限性。}

\StringTok{\#\# 主要发现与结果}

\StringTok{概述文章的核心发现和关键数据。}
\StringTok{对图表、统计数据和实验结果进行总结和分析。}
\StringTok{强调研究结果对原始问题的解答和新发现。}

\StringTok{\#\# 讨论与结论}

\StringTok{分析作者如何讨论结果及其对研究领域的影响。}
\StringTok{总结文章的结论,并指出研究局限性、未解决的问题或作者提出的未来研究方向。}

\StringTok{\#\# 创新点与贡献}

\StringTok{指出文献在理论、方法或实践方面的创新与独特贡献。}
\StringTok{讨论该研究如何推动领域的发展,及其实际应用意义。}

\StringTok{\#\# 个人评价与启示}

\StringTok{提出对文献整体质量、方法合理性和结果可信度的评价。}
\StringTok{指出文献中的不足或存在争议之处。}
\StringTok{给出自己对相关领域未来研究的建议和启示。}

\StringTok{请确保在解读过程中:}
\StringTok{语言表达准确、逻辑清晰;}
\StringTok{分析内容既关注整体框架也注意细节;}
\StringTok{引用和解释关键概念和数据时要做到充分且有条理。}

\StringTok{注意:在输出列表的时候,需要再列表头与列表项之间加入两个空行(换行符),否则Quarto渲染时候会出错。}
\StringTok{"""}

\CommentTok{\# ========== 核心函数:pdf2md ==========}
\KeywordTok{def}\NormalTok{ pdf2md():}
    \CommentTok{\# 读取 json 文件}
    \ControlFlowTok{with} \BuiltInTok{open}\NormalTok{(}\StringTok{"zotero\_update\_summary.json"}\NormalTok{, }\StringTok{"r"}\NormalTok{, encoding}\OperatorTok{=}\StringTok{"utf{-}8"}\NormalTok{) }\ImportTok{as}\NormalTok{ f:}
\NormalTok{        summary }\OperatorTok{=}\NormalTok{ json.load(f)}

    \CommentTok{\# 创建 md 输出目录}
\NormalTok{    md\_output\_dir }\OperatorTok{=}\NormalTok{ os.path.join(os.getcwd(), }\StringTok{"md"}\NormalTok{)}
\NormalTok{    os.makedirs(md\_output\_dir, exist\_ok}\OperatorTok{=}\VariableTok{True}\NormalTok{)}

    \CommentTok{\# 处理新增项}
    \ControlFlowTok{for}\NormalTok{ entry }\KeywordTok{in}\NormalTok{ summary.get(}\StringTok{"新增"}\NormalTok{, []):}
        \ControlFlowTok{for}\NormalTok{ folder\_name }\KeywordTok{in}\NormalTok{ entry:}
\NormalTok{            folder\_path }\OperatorTok{=}\NormalTok{ os.path.join(Zotero\_Storage, folder\_name)}
            \ControlFlowTok{if} \KeywordTok{not}\NormalTok{ os.path.exists(folder\_path):}
                \BuiltInTok{print}\NormalTok{(}\SpecialStringTok{f" 文件夹不存在:}\SpecialCharTok{\{}\NormalTok{folder\_path}\SpecialCharTok{\}}\SpecialStringTok{"}\NormalTok{)}
                \ControlFlowTok{continue}

            \CommentTok{\# 查找 PDF 文件}
\NormalTok{            pdf\_files }\OperatorTok{=}\NormalTok{ [}
\NormalTok{                os.path.join(folder\_path, f)}
                \ControlFlowTok{for}\NormalTok{ f }\KeywordTok{in}\NormalTok{ os.listdir(folder\_path)}
                \ControlFlowTok{if}\NormalTok{ f.lower().endswith(}\StringTok{".pdf"}\NormalTok{)}
\NormalTok{            ]}
            \ControlFlowTok{if} \KeywordTok{not}\NormalTok{ pdf\_files:}
                \BuiltInTok{print}\NormalTok{(}\SpecialStringTok{f" 未找到 PDF 文件:}\SpecialCharTok{\{}\NormalTok{folder\_path}\SpecialCharTok{\}}\SpecialStringTok{"}\NormalTok{)}
                \ControlFlowTok{continue}

\NormalTok{            pdf\_path }\OperatorTok{=}\NormalTok{ pdf\_files[}\DecValTok{0}\NormalTok{]}
            \BuiltInTok{print}\NormalTok{(}\SpecialStringTok{f"}\CharTok{\textbackslash{}n}\SpecialStringTok{ 开始处理:}\SpecialCharTok{\{}\NormalTok{folder\_name}\SpecialCharTok{\}}\SpecialStringTok{"}\NormalTok{)}

            \CommentTok{\# 上传 PDF 文件}
            \ControlFlowTok{try}\NormalTok{:}
\NormalTok{                file\_obj }\OperatorTok{=}\NormalTok{ client.files.create(}\BuiltInTok{file}\OperatorTok{=}\NormalTok{Path(pdf\_path), purpose}\OperatorTok{=}\StringTok{"file{-}extract"}\NormalTok{)}
\NormalTok{                file\_id }\OperatorTok{=}\NormalTok{ file\_obj.}\BuiltInTok{id}
                \BuiltInTok{print}\NormalTok{(}\SpecialStringTok{f" file{-}id:}\SpecialCharTok{\{}\NormalTok{file\_id}\SpecialCharTok{\}}\SpecialStringTok{"}\NormalTok{)}
            \ControlFlowTok{except} \PreprocessorTok{Exception} \ImportTok{as}\NormalTok{ e:}
                \BuiltInTok{print}\NormalTok{(}\SpecialStringTok{f" 上传失败:}\SpecialCharTok{\{}\NormalTok{pdf\_path}\SpecialCharTok{\}}\SpecialStringTok{,错误:}\SpecialCharTok{\{}\NormalTok{e}\SpecialCharTok{\}}\SpecialStringTok{"}\NormalTok{)}
                \ControlFlowTok{continue}

            \CommentTok{\# 发送解析请求}
            \BuiltInTok{print}\NormalTok{(}\StringTok{" 正在发送请求解析文档..."}\NormalTok{)}
            \ControlFlowTok{try}\NormalTok{:}
\NormalTok{                completion }\OperatorTok{=}\NormalTok{ client.chat.completions.create(}
\NormalTok{                    model}\OperatorTok{=}\StringTok{"qwen{-}long"}\NormalTok{,}
\NormalTok{                    messages}\OperatorTok{=}\NormalTok{[}
\NormalTok{                        \{}\StringTok{"role"}\NormalTok{: }\StringTok{"system"}\NormalTok{, }\StringTok{"content"}\NormalTok{: }\SpecialStringTok{f"fileid://}\SpecialCharTok{\{}\NormalTok{file\_id}\SpecialCharTok{\}}\SpecialStringTok{"}\NormalTok{\},}
\NormalTok{                        \{}\StringTok{\textquotesingle{}role\textquotesingle{}}\NormalTok{: }\StringTok{\textquotesingle{}user\textquotesingle{}}\NormalTok{, }\StringTok{\textquotesingle{}content\textquotesingle{}}\NormalTok{: PROMPT\_TEMPLATE\},}
\NormalTok{                    ],}
\NormalTok{                    temperature}\OperatorTok{=}\FloatTok{0.2}\NormalTok{,}
\NormalTok{                )}
\NormalTok{                full\_content }\OperatorTok{=}\NormalTok{ completion.choices[}\DecValTok{0}\NormalTok{].message.content}
            \ControlFlowTok{except} \PreprocessorTok{Exception} \ImportTok{as}\NormalTok{ e:}
                \BuiltInTok{print}\NormalTok{(}\SpecialStringTok{f" 解析失败:}\SpecialCharTok{\{}\NormalTok{folder\_name}\SpecialCharTok{\}}\SpecialStringTok{,错误:}\SpecialCharTok{\{}\NormalTok{e}\SpecialCharTok{\}}\SpecialStringTok{"}\NormalTok{)}
                \ControlFlowTok{continue}

            \CommentTok{\# 保存为 Markdown}
\NormalTok{            md\_path }\OperatorTok{=}\NormalTok{ os.path.join(md\_output\_dir, }\SpecialStringTok{f"}\SpecialCharTok{\{}\NormalTok{folder\_name}\SpecialCharTok{\}}\SpecialStringTok{.md"}\NormalTok{)}
            \ControlFlowTok{with} \BuiltInTok{open}\NormalTok{(md\_path, }\StringTok{"w"}\NormalTok{, encoding}\OperatorTok{=}\StringTok{"utf{-}8"}\NormalTok{) }\ImportTok{as}\NormalTok{ f:}
\NormalTok{                f.write(full\_content)}

            \BuiltInTok{print}\NormalTok{(}\SpecialStringTok{f" 解析成功并保存为 Markdown:}\SpecialCharTok{\{}\NormalTok{md\_path}\SpecialCharTok{\}}\SpecialStringTok{"}\NormalTok{)}

\CommentTok{\# ========== 启动 ==========}
\ControlFlowTok{if} \VariableTok{\_\_name\_\_} \OperatorTok{==} \StringTok{"\_\_main\_\_"}\NormalTok{:}
\NormalTok{    pdf2md()}
\end{Highlighting}
\end{Shaded}

\section{输出结果}\label{ux8f93ux51faux7ed3ux679c}

\begin{Shaded}
\begin{Highlighting}[]
\ImportTok{import}\NormalTok{ yaml}
\ImportTok{import}\NormalTok{ json}
\ImportTok{import}\NormalTok{ os}

\CommentTok{\# 路径配置}
\NormalTok{YML\_PATH }\OperatorTok{=} \StringTok{"\_quarto.yml"}
\NormalTok{STATE\_PATH }\OperatorTok{=} \StringTok{"zotero\_state.json"}
\NormalTok{MD\_FOLDER }\OperatorTok{=} \StringTok{"md"}  \CommentTok{\# md 文件夹}

\CommentTok{\# 读取 \_quarto.yml}
\ControlFlowTok{with} \BuiltInTok{open}\NormalTok{(YML\_PATH, }\StringTok{"r"}\NormalTok{, encoding}\OperatorTok{=}\StringTok{"utf{-}8"}\NormalTok{) }\ImportTok{as}\NormalTok{ f:}
\NormalTok{    yml\_data }\OperatorTok{=}\NormalTok{ yaml.safe\_load(f)}

\CommentTok{\# 确保 book 部分及 chapters 存在}
\NormalTok{yml\_data.setdefault(}\StringTok{"book"}\NormalTok{, \{\}).setdefault(}\StringTok{"chapters"}\NormalTok{, [])}

\CommentTok{\# 读取 zotero\_state.json}
\ControlFlowTok{with} \BuiltInTok{open}\NormalTok{(STATE\_PATH, }\StringTok{"r"}\NormalTok{, encoding}\OperatorTok{=}\StringTok{"utf{-}8"}\NormalTok{) }\ImportTok{as}\NormalTok{ f:}
\NormalTok{    zotero\_state }\OperatorTok{=}\NormalTok{ json.load(f)}

\CommentTok{\# 建立 collection {-}\textgreater{} [md 文件相对路径] 映射}
\NormalTok{collection\_map }\OperatorTok{=}\NormalTok{ \{\}}
\ControlFlowTok{for}\NormalTok{ md\_filename }\KeywordTok{in}\NormalTok{ os.listdir(MD\_FOLDER):}
    \ControlFlowTok{if}\NormalTok{ md\_filename.endswith(}\StringTok{".md"}\NormalTok{):}
\NormalTok{        stem }\OperatorTok{=}\NormalTok{ os.path.splitext(md\_filename)[}\DecValTok{0}\NormalTok{]  }\CommentTok{\# 去掉 .md 后缀}
        \ControlFlowTok{if}\NormalTok{ stem }\KeywordTok{in}\NormalTok{ zotero\_state:}
\NormalTok{            collection }\OperatorTok{=}\NormalTok{ zotero\_state[stem][}\StringTok{"collection"}\NormalTok{]}
            \CommentTok{\# 将相对路径构造为:"md/xxx.md"}
\NormalTok{            md\_path }\OperatorTok{=} \SpecialStringTok{f"}\SpecialCharTok{\{}\NormalTok{MD\_FOLDER}\SpecialCharTok{\}}\SpecialStringTok{/}\SpecialCharTok{\{}\NormalTok{md\_filename}\SpecialCharTok{\}}\SpecialStringTok{"}
\NormalTok{            collection\_map.setdefault(collection, []).append(md\_path)}

\CommentTok{\# 更新 yml\_data,插入 md 文件到对应的 part 下}
\ControlFlowTok{for}\NormalTok{ collection, md\_files }\KeywordTok{in}\NormalTok{ collection\_map.items():}
    \CommentTok{\# 查找是否已有对应的 part}
\NormalTok{    part\_exists }\OperatorTok{=} \VariableTok{False}
    \ControlFlowTok{for}\NormalTok{ item }\KeywordTok{in}\NormalTok{ yml\_data[}\StringTok{"book"}\NormalTok{][}\StringTok{"chapters"}\NormalTok{]:}
        \ControlFlowTok{if} \BuiltInTok{isinstance}\NormalTok{(item, }\BuiltInTok{dict}\NormalTok{) }\KeywordTok{and}\NormalTok{ item.get(}\StringTok{"part"}\NormalTok{) }\OperatorTok{==}\NormalTok{ collection:}
\NormalTok{            part\_exists }\OperatorTok{=} \VariableTok{True}
            \CommentTok{\# 如果已有 part,直接插入新章节(避免重复)}
            \ControlFlowTok{for}\NormalTok{ md\_file }\KeywordTok{in}\NormalTok{ md\_files:}
                \ControlFlowTok{if}\NormalTok{ md\_file }\KeywordTok{not} \KeywordTok{in}\NormalTok{ item[}\StringTok{"chapters"}\NormalTok{]:}
\NormalTok{                    item[}\StringTok{"chapters"}\NormalTok{].append(md\_file)}
            \ControlFlowTok{break}

    \CommentTok{\# 如果没有对应的 part,则创建新的 part 和章节}
    \ControlFlowTok{if} \KeywordTok{not}\NormalTok{ part\_exists:}
\NormalTok{        new\_part }\OperatorTok{=}\NormalTok{ \{}\StringTok{"part"}\NormalTok{: collection, }\StringTok{"chapters"}\NormalTok{: md\_files\}}
\NormalTok{        yml\_data[}\StringTok{"book"}\NormalTok{][}\StringTok{"chapters"}\NormalTok{].append(new\_part)}

\CommentTok{\# 写回 \_quarto.yml,确保新的内容仅添加到 book 部分的 chapters 下}
\ControlFlowTok{with} \BuiltInTok{open}\NormalTok{(YML\_PATH, }\StringTok{"w"}\NormalTok{, encoding}\OperatorTok{=}\StringTok{"utf{-}8"}\NormalTok{) }\ImportTok{as}\NormalTok{ f:}
\NormalTok{    yaml.dump(yml\_data, f, allow\_unicode}\OperatorTok{=}\VariableTok{True}\NormalTok{, sort\_keys}\OperatorTok{=}\VariableTok{False}\NormalTok{)}

\BuiltInTok{print}\NormalTok{(}\StringTok{" \_quarto.yml 文件已成功更新!"}\NormalTok{)}
\end{Highlighting}
\end{Shaded}

\section{整合流程}\label{ux6574ux5408ux6d41ux7a0b}

我们使用 \texttt{invoke} 命令来执行自动化任务。所有任务均定义在
\texttt{tasks.py} 中,关键任务及其作用如下:

\begin{Shaded}
\begin{Highlighting}[]
\ImportTok{import}\NormalTok{ os}
\ImportTok{import}\NormalTok{ sys}
\ImportTok{import}\NormalTok{ yaml}
\ImportTok{import}\NormalTok{ re}
\ImportTok{from}\NormalTok{ subprocess }\ImportTok{import}\NormalTok{ check\_output}
\ImportTok{from}\NormalTok{ invoke }\ImportTok{import}\NormalTok{ task}

\KeywordTok{def}\NormalTok{ get\_env\_name(yaml\_file}\OperatorTok{=}\StringTok{"environment.yaml"}\NormalTok{):}
    \CommentTok{"""}
\CommentTok{    通过 PyYAML 从 environment.yaml 文件中获取环境名称。}
\CommentTok{    若文件不存在或未找到 \textquotesingle{}name\textquotesingle{} 键,则输出警告或退出程序。}
\CommentTok{    """}
    \ControlFlowTok{if}\NormalTok{ os.path.exists(yaml\_file):}
        \ControlFlowTok{with} \BuiltInTok{open}\NormalTok{(yaml\_file, }\StringTok{"r"}\NormalTok{, encoding}\OperatorTok{=}\StringTok{"utf{-}8"}\NormalTok{) }\ImportTok{as}\NormalTok{ f:}
            \ControlFlowTok{try}\NormalTok{:}
\NormalTok{                env\_data }\OperatorTok{=}\NormalTok{ yaml.safe\_load(f)}
\NormalTok{                env\_name }\OperatorTok{=}\NormalTok{ env\_data.get(}\StringTok{"name"}\NormalTok{)}
                \ControlFlowTok{if}\NormalTok{ env\_name:}
                    \ControlFlowTok{return}\NormalTok{ env\_name.strip()}
                \ControlFlowTok{else}\NormalTok{:}
                    \BuiltInTok{print}\NormalTok{(}\StringTok{"Warning: environment.yaml 中未找到 \textquotesingle{}name\textquotesingle{} 键。"}\NormalTok{)}
            \ControlFlowTok{except} \PreprocessorTok{Exception} \ImportTok{as}\NormalTok{ e:}
                \BuiltInTok{print}\NormalTok{(}\StringTok{"Error parsing environment.yaml:"}\NormalTok{, e)}
\NormalTok{                sys.exit(}\DecValTok{1}\NormalTok{)}
    \ControlFlowTok{else}\NormalTok{:}
        \BuiltInTok{print}\NormalTok{(}\StringTok{"Warning: environment.yaml 文件不存在。"}\NormalTok{)}
    \ControlFlowTok{return} \VariableTok{None}

\KeywordTok{def}\NormalTok{ conda\_env\_exists(env\_name):}
    \CommentTok{"""}
\CommentTok{    检查 Conda 环境是否存在。}

\CommentTok{    通过调用 "conda env list" 获取已注册的环境信息,}
\CommentTok{    如果输出中显示环境名称为完整路径,则提取最后一部分进行匹配,}
\CommentTok{    同时匹配不区分大小写。}
\CommentTok{    """}
    \ControlFlowTok{try}\NormalTok{:}
\NormalTok{        envs\_output }\OperatorTok{=}\NormalTok{ check\_output(}\StringTok{"conda env list"}\NormalTok{, shell}\OperatorTok{=}\VariableTok{True}\NormalTok{, encoding}\OperatorTok{=}\StringTok{"utf{-}8"}\NormalTok{)}
\NormalTok{        target }\OperatorTok{=}\NormalTok{ env\_name.lower()}
        \ControlFlowTok{for}\NormalTok{ line }\KeywordTok{in}\NormalTok{ envs\_output.splitlines():}
\NormalTok{            line }\OperatorTok{=}\NormalTok{ line.strip()}
            \ControlFlowTok{if} \KeywordTok{not}\NormalTok{ line }\KeywordTok{or}\NormalTok{ line.startswith(}\StringTok{"\#"}\NormalTok{):}
                \ControlFlowTok{continue}
            \CommentTok{\# 去除当前激活环境标记 \textquotesingle{}*\textquotesingle{}}
\NormalTok{            line }\OperatorTok{=}\NormalTok{ line.replace(}\StringTok{"*"}\NormalTok{, }\StringTok{""}\NormalTok{).strip()}
\NormalTok{            match }\OperatorTok{=}\NormalTok{ re.match(}\VerbatimStringTok{r"\^{}(\textbackslash{}S+)"}\NormalTok{, line)}
            \ControlFlowTok{if}\NormalTok{ match:}
\NormalTok{                candidate }\OperatorTok{=}\NormalTok{ match.group(}\DecValTok{1}\NormalTok{).strip()}
                \CommentTok{\# 如果 candidate 是一个路径(包含路径分隔符),则提取 basename}
                \ControlFlowTok{if}\NormalTok{ os.path.sep }\KeywordTok{in}\NormalTok{ candidate:}
\NormalTok{                    candidate }\OperatorTok{=}\NormalTok{ os.path.basename(candidate)}
                \ControlFlowTok{if}\NormalTok{ candidate.lower() }\OperatorTok{==}\NormalTok{ target:}
                    \ControlFlowTok{return} \VariableTok{True}
        \ControlFlowTok{return} \VariableTok{False}
    \ControlFlowTok{except} \PreprocessorTok{Exception} \ImportTok{as}\NormalTok{ e:}
        \BuiltInTok{print}\NormalTok{(}\StringTok{"无法检查 conda 环境:"}\NormalTok{, e)}
        \ControlFlowTok{return} \VariableTok{False}

\AttributeTok{@task}
\KeywordTok{def}\NormalTok{ setup\_env(c):}
    \CommentTok{"""}
\CommentTok{    检查并创建或更新 Conda 环境。}

\CommentTok{      1. 从 environment.yaml 中读取目标环境名称;}
\CommentTok{      2. 如果当前已处于目标环境中,则直接退出;}
\CommentTok{      3. 如果目标环境已经存在,则提示用户手动激活;}
\CommentTok{      4. 如果目标环境不存在,则使用 environment.yaml 自动创建环境。}
\CommentTok{    """}
\NormalTok{    yaml\_file }\OperatorTok{=} \StringTok{"environment.yaml"}
\NormalTok{    env\_name }\OperatorTok{=}\NormalTok{ get\_env\_name(yaml\_file)}
    \ControlFlowTok{if} \KeywordTok{not}\NormalTok{ env\_name:}
        \BuiltInTok{print}\NormalTok{(}\StringTok{"未能从 environment.yaml 获取环境名称。"}\NormalTok{)}
\NormalTok{        sys.exit(}\DecValTok{1}\NormalTok{)}

    \BuiltInTok{print}\NormalTok{(}\SpecialStringTok{f"目标 Conda 环境: }\SpecialCharTok{\{}\NormalTok{env\_name}\SpecialCharTok{\}}\SpecialStringTok{"}\NormalTok{)}

    \CommentTok{\# 检查当前是否已在目标环境中(忽略大小写)}
\NormalTok{    current\_env }\OperatorTok{=}\NormalTok{ os.environ.get(}\StringTok{"CONDA\_DEFAULT\_ENV"}\NormalTok{)}
    \ControlFlowTok{if}\NormalTok{ current\_env }\KeywordTok{and}\NormalTok{ current\_env.lower() }\OperatorTok{==}\NormalTok{ env\_name.lower():}
        \BuiltInTok{print}\NormalTok{(}\StringTok{"当前已在目标 Conda 环境中。"}\NormalTok{)}
        \ControlFlowTok{return}

    \ControlFlowTok{if}\NormalTok{ conda\_env\_exists(env\_name):}
        \BuiltInTok{print}\NormalTok{(}\SpecialStringTok{f"环境 \textquotesingle{}}\SpecialCharTok{\{}\NormalTok{env\_name}\SpecialCharTok{\}}\SpecialStringTok{\textquotesingle{} 已存在。"}\NormalTok{)}
        \BuiltInTok{print}\NormalTok{(}\SpecialStringTok{f"请在命令行中运行: conda activate }\SpecialCharTok{\{}\NormalTok{env\_name}\SpecialCharTok{\}}\SpecialStringTok{"}\NormalTok{)}
        \BuiltInTok{print}\NormalTok{(}\StringTok{"激活后再重新运行此命令。"}\NormalTok{)}
\NormalTok{        sys.exit(}\DecValTok{0}\NormalTok{)}
    \ControlFlowTok{else}\NormalTok{:}
        \BuiltInTok{print}\NormalTok{(}\SpecialStringTok{f"环境 \textquotesingle{}}\SpecialCharTok{\{}\NormalTok{env\_name}\SpecialCharTok{\}}\SpecialStringTok{\textquotesingle{} 不存在,正在创建..."}\NormalTok{)}
        \ControlFlowTok{try}\NormalTok{:}
\NormalTok{            result }\OperatorTok{=}\NormalTok{ c.run(}\SpecialStringTok{f"conda env create {-}f }\SpecialCharTok{\{}\NormalTok{yaml\_file}\SpecialCharTok{\}}\SpecialStringTok{"}\NormalTok{, warn}\OperatorTok{=}\VariableTok{True}\NormalTok{)}
            \ControlFlowTok{if}\NormalTok{ result.failed:}
                \BuiltInTok{print}\NormalTok{(}\StringTok{"创建环境失败,请手动处理。"}\NormalTok{)}
\NormalTok{                sys.exit(}\DecValTok{1}\NormalTok{)}
            \ControlFlowTok{else}\NormalTok{:}
                \BuiltInTok{print}\NormalTok{(}\SpecialStringTok{f"环境创建成功!请运行: conda activate }\SpecialCharTok{\{}\NormalTok{env\_name}\SpecialCharTok{\}}\SpecialStringTok{"}\NormalTok{)}
\NormalTok{                sys.exit(}\DecValTok{0}\NormalTok{)}
        \ControlFlowTok{except} \PreprocessorTok{Exception} \ImportTok{as}\NormalTok{ e:}
            \BuiltInTok{print}\NormalTok{(}\StringTok{"创建环境失败,请手动处理:"}\NormalTok{, e)}
\NormalTok{            sys.exit(}\DecValTok{1}\NormalTok{)}

\ControlFlowTok{if} \VariableTok{\_\_name\_\_} \OperatorTok{==} \StringTok{"\_\_main\_\_"}\NormalTok{:}
    \ImportTok{from}\NormalTok{ invoke }\ImportTok{import}\NormalTok{ Program}
\NormalTok{    program }\OperatorTok{=}\NormalTok{ Program(namespace}\OperatorTok{=}\BuiltInTok{globals}\NormalTok{())}
\NormalTok{    program.run()}



\AttributeTok{@task}
\KeywordTok{def}\NormalTok{ run\_paper(c):}
    \BuiltInTok{print}\NormalTok{(}\StringTok{"正在运行 paper.py..."}\NormalTok{)}
\NormalTok{    c.run(}\StringTok{"python paper.py"}\NormalTok{)    }

\AttributeTok{@task}
\KeywordTok{def}\NormalTok{ run\_zotero\_update(c):}
    \BuiltInTok{print}\NormalTok{(}\StringTok{"正在运行 zotero\_update.py..."}\NormalTok{)}
\NormalTok{    c.run(}\StringTok{"python zotero\_update.py"}\NormalTok{)}

\AttributeTok{@task}
\KeywordTok{def}\NormalTok{ run\_pdf2md(c):}
    \BuiltInTok{print}\NormalTok{(}\StringTok{"正在运行 pdf2md.py..."}\NormalTok{)}
\NormalTok{    c.run(}\StringTok{"python pdf2md.py"}\NormalTok{)}

\AttributeTok{@task}
\KeywordTok{def}\NormalTok{ run\_add2yml(c):}
    \BuiltInTok{print}\NormalTok{(}\StringTok{"正在运行 add2yml.py..."}\NormalTok{)}
\NormalTok{    c.run(}\StringTok{"python add2yml.py"}\NormalTok{)}

\AttributeTok{@task}
\KeywordTok{def}\NormalTok{ run\_quarto(c):}
    \BuiltInTok{print}\NormalTok{(}\StringTok{"正在执行 quarto render..."}\NormalTok{)}
\NormalTok{    c.run(}\StringTok{"quarto render"}\NormalTok{)}

\AttributeTok{@task}\NormalTok{(pre}\OperatorTok{=}\NormalTok{[setup}\OperatorTok{{-}}\NormalTok{env, run\_zotero\_update, run\_pdf2md, run\_add2yml, run\_quarto])}
\KeywordTok{def} \BuiltInTok{all}\NormalTok{(c):}
    \BuiltInTok{print}\NormalTok{(}\StringTok{"所有任务执行完成!"}\NormalTok{)}
\end{Highlighting}
\end{Shaded}

\bookmarksetup{startatroot}

\chapter{如何使用}\label{ux5982ux4f55ux4f7fux7528}

\section{环境准备}\label{ux73afux5883ux51c6ux5907}

请按以下顺序确保系统满足要求:

\subsection{安装 Python 与 pip}\label{ux5b89ux88c5-python-ux4e0e-pip}

\begin{itemize}
\item
  \textbf{要求}:安装 Python 3.x 与 pip。\\
\item
  \textbf{获取方式}:\href{https://www.python.org/downloads/}{Python
  官网}\\
\item
  \textbf{验证方法}:

\begin{Shaded}
\begin{Highlighting}[]
\ExtensionTok{python} \AttributeTok{{-}{-}version}
\ExtensionTok{pip} \AttributeTok{{-}{-}version}
\end{Highlighting}
\end{Shaded}
\end{itemize}

\subsection{安装 Conda
环境管理工具}\label{ux5b89ux88c5-conda-ux73afux5883ux7ba1ux7406ux5de5ux5177}

\begin{itemize}
\tightlist
\item
  \textbf{要求}:安装 Anaconda 或
  Miniconda,用于创建和管理独立的开发环境。\\
\item
  \textbf{获取方式}:

  \begin{itemize}
  \tightlist
  \item
    \href{https://www.anaconda.com/products/individual}{Anaconda 官网}\\
  \item
    \href{https://docs.conda.io/en/latest/miniconda.html}{Miniconda
    官网}
  \end{itemize}
\end{itemize}

\subsection{安装 Invoke
与其它依赖库}\label{ux5b89ux88c5-invoke-ux4e0eux5176ux5b83ux4f9dux8d56ux5e93}

在 Python 与 Conda 安装完成后,运行以下命令安装项目依赖:

\begin{Shaded}
\begin{Highlighting}[]
\ExtensionTok{pip}\NormalTok{ install invoke pyyaml}
\end{Highlighting}
\end{Shaded}

\subsection{安装 Quarto(可选)}\label{ux5b89ux88c5-quartoux53efux9009}

\begin{itemize}
\item
  \textbf{要求}:若需执行文档渲染任务,必须安装 Quarto。\\
\item
  \textbf{获取方式}:\href{https://quarto.org/}{Quarto 官网}\\
\item
  \textbf{验证方法}:

\begin{Shaded}
\begin{Highlighting}[]
\ExtensionTok{quarto}\NormalTok{ render}
\end{Highlighting}
\end{Shaded}
\end{itemize}

\section{配置文件说明}\label{ux914dux7f6eux6587ux4ef6ux8bf4ux660e}

项目使用两个配置文件,各有不同作用:

\subsection{env.yml(密钥与 API
参数配置)}\label{env.ymlux5bc6ux94a5ux4e0e-api-ux53c2ux6570ux914dux7f6e}

\begin{itemize}
\item
  \textbf{作用}:存放密钥、API 参数等敏感信息。\\
\item
  \textbf{示例配置}:

\begin{Shaded}
\begin{Highlighting}[]
\CommentTok{\# env.yml 配置示例}
\FunctionTok{Zotero\_Storage}\KeywordTok{:}\AttributeTok{ }\StringTok{"C:/Path/To/Your/Zotero/Storage"}
\FunctionTok{Bai\_Lian\_API\_KEY}\KeywordTok{:}\AttributeTok{ }\StringTok{"\textless{}your\_bai\_lian\_api\_key\textgreater{}"}
\FunctionTok{Zotero\_user\_id}\KeywordTok{:}\AttributeTok{ }\StringTok{"\textless{}your\_zotero\_user\_id\textgreater{}"}
\FunctionTok{Zotero\_API\_KEY}\KeywordTok{:}\AttributeTok{ }\StringTok{"\textless{}your\_zotero\_api\_key\textgreater{}"}
\FunctionTok{Zotero\_BASE\_URL}\KeywordTok{:}\AttributeTok{ }\StringTok{"https://api.zotero.org"}
\end{Highlighting}
\end{Shaded}
\end{itemize}

\subsection{environment.yaml(Conda
环境配置)}\label{environment.yamlconda-ux73afux5883ux914dux7f6e}

\begin{itemize}
\item
  \textbf{作用}:定义项目开发和运行所需的 Conda 环境。\\
\item
  \textbf{示例配置}:

\begin{Shaded}
\begin{Highlighting}[]
\CommentTok{\# environment.yaml 配置示例}
\FunctionTok{name}\KeywordTok{:}\AttributeTok{ my\_project\_env}
\FunctionTok{channels}\KeywordTok{:}
\AttributeTok{  }\KeywordTok{{-}}\AttributeTok{ defaults}
\FunctionTok{dependencies}\KeywordTok{:}
\AttributeTok{  }\KeywordTok{{-}}\AttributeTok{ python=3.x}
\AttributeTok{  }\KeywordTok{{-}}\AttributeTok{ pip}
\AttributeTok{  }\KeywordTok{{-}}\AttributeTok{ pyyaml}
\AttributeTok{  }\KeywordTok{{-}}\AttributeTok{ invoke}
\CommentTok{  \# 根据需要添加其它依赖}
\end{Highlighting}
\end{Shaded}
\end{itemize}

在执行自动化任务时,\texttt{setup-env} 将根据 environment.yaml
的配置检测当前环境,若不匹配或不存在,则自动创建新环境。

\section{自动化任务说明}\label{ux81eaux52a8ux5316ux4efbux52a1ux8bf4ux660e}

所有任务均定义在 \texttt{tasks.py} 中,关键任务及其作用如下:

\subsection{setup-env}\label{setup-env}

\begin{itemize}
\tightlist
\item
  \textbf{作用}:检查当前 Conda 环境是否符合 environment.yaml 中的配置。

  \begin{itemize}
  \tightlist
  \item
    环境已激活:提示环境正确。\\
  \item
    环境存在但未激活:提示执行
    \texttt{conda\ activate\ \textless{}env\_name\textgreater{}}。\\
  \item
    环境不存在:执行 \texttt{conda\ env\ create\ -f\ environment.yaml}
    自动创建。
  \end{itemize}
\end{itemize}

\subsection{run-paper}\label{run-paper}

\begin{itemize}
\tightlist
\item
  \textbf{作用}:执行 \texttt{paper.py}
  脚本,用于论文检索与筛选(无相关需求可跳过)。
\end{itemize}

\subsection{run-zotero-update}\label{run-zotero-update}

\begin{itemize}
\tightlist
\item
  \textbf{作用}:执行 \texttt{zotero-update.py} 脚本,主要任务:

  \begin{itemize}
  \tightlist
  \item
    更新 Zotero 数据,并实时维护文献状态。

    \begin{itemize}
    \tightlist
    \item
      更新后的文献信息会保存至 \textbf{state.json},该文件用于存储
      Zotero 当前状态数据,例如已同步的文献条目和相关更新信息;\\
    \end{itemize}
  \item
    输出更新记录至
    \textbf{zotero\_update\_summary.json},该文件详细记录每次更新的摘要和变动信息,便于后续任务参考。
  \end{itemize}
\end{itemize}

\subsection{run-pdf2md}\label{run-pdf2md}

\begin{itemize}
\tightlist
\item
  \textbf{作用}:执行 \texttt{pdf2md.py} 脚本,该任务利用
  \textbf{zotero\_update\_summary.json} 中的更新记录:

  \begin{itemize}
  \tightlist
  \item
    读取文件中的更新记录,获取需解析的 PDF 文件信息;\\
  \item
    将这些 PDF 文件提交给 AI 模块进行解析,生成 Markdown
    格式的解读报告;\\
  \item
    输出的 Markdown 报告用于后续整合与渲染。
  \end{itemize}
\end{itemize}

\subsection{run-add2yml}\label{run-add2yml}

\begin{itemize}
\tightlist
\item
  \textbf{作用}:执行 \texttt{add2yml.py}
  脚本,将生成的解读报告数据添加至 YAML 文件中,便于记录与后续处理。
\end{itemize}

\subsection{run-quarto}\label{run-quarto}

\begin{itemize}
\tightlist
\item
  \textbf{作用}:调用 \texttt{quarto\ render}
  命令,渲染生成的解读报告(前提是已正确安装 Quarto)。
\end{itemize}

\subsection{all (一键部署)}\label{all-ux4e00ux952eux90e8ux7f72}

\begin{itemize}
\tightlist
\item
  \textbf{作用}:按以下顺序依次执行所有任务:

  \begin{enumerate}
  \def\labelenumi{\arabic{enumi}.}
  \tightlist
  \item
    setup-env
  \item
    run-zotero-update\\
  \item
    run-pdf2md\\
  \item
    run-add2yml\\
  \item
    run-quarto\\
  \end{enumerate}
\item
  执行完成后终端显示``所有任务执行完成!''的提示。
\end{itemize}

\emph{任务执行命令格式统一为:}

\begin{Shaded}
\begin{Highlighting}[]
\ExtensionTok{invoke} \OperatorTok{\textless{}}\NormalTok{任务名}\OperatorTok{\textgreater{}}
\end{Highlighting}
\end{Shaded}

\section{详细运行步骤}\label{ux8be6ux7ec6ux8fd0ux884cux6b65ux9aa4}

按以下步骤操作,确保项目任务顺利执行:

\subsection{打开终端}\label{ux6253ux5f00ux7ec8ux7aef}

\begin{itemize}
\tightlist
\item
  Windows 用户:使用命令提示符或 PowerShell;\\
\item
  macOS/Linux 用户:使用 Terminal。
\end{itemize}

\subsection{切换到项目根目录}\label{ux5207ux6362ux5230ux9879ux76eeux6839ux76eeux5f55}

确保 \texttt{tasks.py}、\texttt{env.yml} 和 \texttt{environment.yaml}
均在同一目录中。

示例命令:

\begin{Shaded}
\begin{Highlighting}[]
\BuiltInTok{cd}\NormalTok{ /你的/项目/所在/目录}
\end{Highlighting}
\end{Shaded}

\subsection{执行各任务}\label{ux6267ux884cux5404ux4efbux52a1}

\begin{enumerate}
\def\labelenumi{\arabic{enumi}.}
\item
  \textbf{检查或创建 Conda 环境}

  运行命令:

\begin{Shaded}
\begin{Highlighting}[]
\ExtensionTok{invoke}\NormalTok{ setup{-}env}
\end{Highlighting}
\end{Shaded}

  \begin{itemize}
  \tightlist
  \item
    若未激活已创建环境,请根据提示执行
    \texttt{conda\ activate\ \textless{}env\_name\textgreater{}}。
  \end{itemize}
\item
  \textbf{更新 Zotero 数据}

  运行命令:

\begin{Shaded}
\begin{Highlighting}[]
\ExtensionTok{invoke}\NormalTok{ run{-}zotero{-}update}
\end{Highlighting}
\end{Shaded}

  \begin{itemize}
  \tightlist
  \item
    执行后检查生成的两个文件:

    \begin{itemize}
    \tightlist
    \item
      \textbf{state.json}:存储 Zotero
      最新同步状态信息(如文献条目、更新时间等);\\
    \item
      \textbf{zotero\_update\_summary.json}:记录更新摘要与变动详情,供后续任务使用。
    \end{itemize}
  \end{itemize}
\item
  \textbf{解析 PDF 并生成 Markdown 报告}

  运行命令:

\begin{Shaded}
\begin{Highlighting}[]
\ExtensionTok{invoke}\NormalTok{ run{-}pdf2md}
\end{Highlighting}
\end{Shaded}

  \begin{itemize}
  \tightlist
  \item
    脚本从 \textbf{zotero\_update\_summary.json}
    中读取更新记录,获取需解析的 PDF 文件信息。\\
  \item
    AI 模块解析后生成 Markdown 格式的报告,供后续编辑和使用。
  \end{itemize}
\item
  \textbf{将解读报告数据添加至 YAML 文件}

  运行命令:

\begin{Shaded}
\begin{Highlighting}[]
\ExtensionTok{invoke}\NormalTok{ run{-}add2yml}
\end{Highlighting}
\end{Shaded}
\item
  \textbf{渲染解读报告(可选)}

  若需转换 Markdown 报告为最终格式,运行命令:

\begin{Shaded}
\begin{Highlighting}[]
\ExtensionTok{invoke}\NormalTok{ run{-}quarto}
\end{Highlighting}
\end{Shaded}

  \begin{itemize}
  \tightlist
  \item
    请确保 Quarto 已正确安装和配置。
  \end{itemize}
\item
  \textbf{整体执行所有任务(除检索文件的所有步骤)}

  一步完成整个流程,运行命令:

\begin{Shaded}
\begin{Highlighting}[]
\ExtensionTok{invoke}\NormalTok{ all}
\end{Highlighting}
\end{Shaded}

  \begin{itemize}
  \tightlist
  \item
    系统将按顺序执行所有任务,并在任务结束时显示成功提示。
  \end{itemize}
\end{enumerate}

\section{附:state.json 与 zotero\_update\_summary.json
说明}\label{ux9644state.json-ux4e0e-zotero_update_summary.json-ux8bf4ux660e}

\begin{itemize}
\tightlist
\item
  \textbf{state.json}

  \begin{itemize}
  \tightlist
  \item
    用于记录 Zotero 数据的当前状态。\\
  \item
    保存同步时刻的文献条目列表、标记信息等,确保后续更新时可以对比和判断变动。
  \end{itemize}
\item
  \textbf{zotero\_update\_summary.json}

  \begin{itemize}
  \tightlist
  \item
    用于记录每次运行 \texttt{zotero-update.py} 时的更新内容。\\
  \item
    包含新增文献、其所在目录的关键数据,供 \textbf{run-pdf2md}
    任务参考,确保仅解析最新更新的 PDF 文件信息。
  \end{itemize}
\end{itemize}

\bookmarksetup{startatroot}

\chapter{示例结果}\label{ux793aux4f8bux7ed3ux679c}

这一部分,我们将采集关键词 ``\,'' 的文献 3 篇,并进行结果展示。

\section{操作流程}\label{ux64cdux4f5cux6d41ux7a0b}

\section{结果展示}\label{ux7ed3ux679cux5c55ux793a}




\end{document}
